\chapter{Scope}

Identify an upcoming deal making / negotiation opportunity at work (for example, it could be a sale or a contract negotiation or a job interview).
How are you going to prepare to participate in the process? Present a detailed analysis.
Some guidelines (feel free to add further details): \\

\begin{itemize}
	\item Situation analysis must be clearly documented
	\item Identify with reasoning the appropriate style of negotiation you will choose.
	\item Who are the negotiating parties?
	\item What are they negotiating about / what are they going to negotiate about?
	\item What is your role in the negotiation?
	\item What is your BATNA?
	\item What is or could be the BATNA of the other party / parties?
	\item What learning in this course so far can you apply in the preparation?
\end{itemize}



\chapter{use case}

The specific use case (deal making / negotiation opportunity) chosen is the \textbf{job interview}. \\

\noindent A typical job interview negotiation between me as the applicant and 
my possible future employer (or human resources staff) will be discussed. \\
 
\chapter{situation analysis}

Some weeks before, I have applied for a job as a DevOps Engineer at 
a software developing company in Austria. Personnel from the human resources department
contacted me afterwards, and we fixed a date for a job interview. The interview is going to
be held in person at the headquarters of the company in Vienna. The staff member
from the human resources department, as well as my possible future team leader
will be interviewing me. \\

\noindent Within the selection of the word interview lies the first problem, which needs
to be pointed out. In fact, we should not see the
\emph{job interview} as a classic interview, for which the definition
is the following: \\

\noindent An interview is a structured conversation where one participant asks questions, and the other provides answers \autocite{merriamWebsterDefinitionInterview}. \\

\noindent Instead the job interview, as part of the job application process,
should be seen as a \emph{negotiation}. \\

\begin{center}
	\noindent Negotiation is a process whereby two or
	more parties work toward an agreement.
\end{center}

% who are the negotiating parties

\noindent The negotiating parties are

\begin{itemize}
	\item job applicant (interviewee)
	\item employer, HR staff (interviewer)
\end{itemize}

% What are they negotiating about / what are they going to negotiate about?

\noindent This negotiation is interest-based. The involved parties want different things.
While I as the job applicant want things like

\begin{itemize}
	\item appropriate salary
	\item benefits
	\item friendly colleagues
\end{itemize}

\noindent the employer wants things like certain skills of the employee

\begin{itemize}
	\item formal qualifications
	\item communication skills
	\item long term commitment
\end{itemize}

% role of applicant in the negotiation

\noindent My role in the negotiation, as the applicant in the job interview,
is primarily to negotiate my key interests with my negotiation partner
(future employer). The employer might try to bring down my hopes and wants.
But I need to use negotiation techniques, in order to not sell myself under
my value. I can definitely not fall below my BATNA. 

\begin{center}
	\noindent Best Alternative To a Negotiated Agreement
\end{center}

% BATNA of applicant

\noindent To show an example, we can set a BATNA for my salary. 
My current salary is 42,000 euros per year (3,000 euros per month).
This is my BATNA. It is the value, which I will fall back to,
when the negotiation fails. \\

% BATNA of other party

\noindent Let's make up a BATNA for my potential employer. 
There is already another applicant, who has similar qualifications as me,
and he assured his negotiation with a salary of 57,000 euros per year.
This is the BATNA of my negotiation partner, the employer. \\

\noindent The lowest salary I am willing to 
accept in the negotiation is 60,000 euros per year (4,285.71 euros per month). \\

% identify style of negotiation one would want to use

\noindent From theory, we know five different negotiation styles

\begin{itemize}
	\item Competing
	\item Avoiding
	\item Compromising
	\item Collaborating
	\item Accommodating
\end{itemize}

\noindent In this kind of negotiation (job interview) from the perspective 
of the applicant (interviewee), I would aim for the \emph{Collaborating} 
negotiation style. This way, both parties win and meet their requirements and
needs as the outcome. However, this requires cooperation from both parties,
and might not always be possible. 


% TODO: What learning in this course so far can you apply in the preparation?





















