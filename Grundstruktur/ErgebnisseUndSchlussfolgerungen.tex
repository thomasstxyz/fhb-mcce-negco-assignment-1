\chapter{Conclusion} \label{sec:ErgeUndSchl}
%Die Ergebnisse der Arbeit sind in übersichtlicher Form darzustellen Die
%gewählte Form der Darstellung ist vom gewählten Datenmaterial und den in der
%Einleitung gesetzten Zielen abhängig. Die Ergebnisse sind zu interpretieren und
%in Bezug zum Stand des Wissens zu diskutieren. Über die Beantwortung der
%Forschungsfrage und die daraus gezogenen Schlussfolgerungen schließt sich der
%Bogen zur Einleitung.

%\noindent Wichtig ist die gedanklich klare Unterscheidung zwischen der
%Darstellung der Ergebnisse und der Interpretation/Bewertung der Ergebnisse. \\







%\section{Conclusion}

%\noindent Zusammenfassend lässt sich sagen, dass der Apache Server als Load Balancer
%eingesetzt werden kann, es sollte nur sichergestellt werden, dass das \emph{MPM event}
%geladen ist. Bei wenigen Anfragen, gemeinsam mit großen Websites, bietet Apache
%eine bessere Leistung als Nginx. Sollen jedoch viele kleine Websites bzw. Dateien
%und eine große Anzahl an Clients bearbeitet werden, so liefert Nginx eine durchaus
%bessere Performance als Apache.







%


