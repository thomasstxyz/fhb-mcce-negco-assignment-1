\chapter{Approach and methods} \label{sec:Vorgangsweise}
%Die Forschungsfrage kann nicht mit einer reinen Literaturarbeit untersucht
%und beantwortet werden, da nur wenig spezifische Literatur gefunden und
%diese auch nur als Basis verwendet werden konnte. Die genauere Untersuchung
%bedarf einer komplexeren Forschungsmethode, im konkreten Fall das technische
%Experiment, welches innerhalb einer Laborumgebung durchgeführt wird.
%Nach Durchführung des Experiments erfolgt eine objektive Darstellung der
%Ergebnisse, die im Anschluss statistisch analysiert wird.
%Zum Schluss werden Schlussfolgerungen und Empfehlungen ausgesagt.

\section{method}
%Die eingesetzte Methoden (z.B. Online-Befragung, Inhaltsanalyse, Interviews) müssen ebenfalls nachvollziehbar beschrieben werden.
%Unterschiedliche Untersuchungsmethoden haben oft unterschiedliche Genauigkeit.
%Neben der Begründung und Beschreibung der Untersuchungsmethoden ist
%auch eine Begründung und Beschreibung der verwendeten Auswertungsmethoden bzw.
%dafür verwendete Software unerlässlich. \\

%\footnote{Wenn der Abstand zwischen Fußnotentrennstrich und Fußnote zu groß wird, gehen Sie folgend vor:

%\noindent Wählen Sie im Hauptmenü „Ansicht $|$ Entwurf $|$ Verweise $|$ Notizen anzeigen $|$ Fußnotentrennlinie". Dann können Sie unnötige Leerzeichen entfernen.

%}

%\subsection{$<<$Überschrift 3. Ebene$>>$}
%
%\subsubsection{$<<$Überschrift 4. Ebene$>>$}
%4 Überschriftebenen müssen reichen.
%
%\subsection{$<<$Überschrift 3. Ebene$>>$}
%Wenn es ein Kapitel 3.2.1 gibt, muss es auch ein Kapitel 3.2.2 geben.



%Im Zuge dieser Arbeit wird die zentrale Forschungsfrage mittels
%eines technischen Experiments erforscht.

%\subsection{technisches Experiment}
