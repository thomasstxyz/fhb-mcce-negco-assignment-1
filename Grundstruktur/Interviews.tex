\documentclass[fontsize=11pt,paper=a4]{scrbook}
\usepackage{calc}
\usepackage[ngerman]{babel}
\usepackage[T1]{fontenc}
\usepackage[utf8]{inputenc}
\usepackage{setspace}
\usepackage{booktabs,multirow,tabularx}
\usepackage{ulem}
\usepackage{libertine}
%---------------------------------------------------------------------------
\usepackage[pagewise]{lineno}
\def\linenumberfont{\normalfont\small}
\usepackage[htt]{hyphenat} % Trennung von Typewriter Fonts
%---------------------------------------------------------------------------
% Full justification with typewriter font
%---------------------------------------------------------------------------
\usepackage{everysel}
\EverySelectfont{%
\fontdimen2\font=0.4em% interword space
\fontdimen3\font=0.2em% interword stretch
\fontdimen4\font=0.1em% interword shrink
\fontdimen7\font=0.1em% extra space
\hyphenchar\font=`\-% to allow hyphenation
}
%---------------------------------------------------------------------------
% Abstände zischen items verringern
%---------------------------------------------------------------------------
\usepackage{mdwlist}
%---------------------------------------------------------------------------
\begin{document}

\chapter*{Transkriptionen der Interviews}
\section*{Interview 1} 
\texttt{
\begin{itemize*} 
\rightlinenumbers*
\modulolinenumbers[5] 
\linenumbers[0] 
\item[AS:] Wie viele Jahre unterrichtest du schon?
\item[IP1:] 14 Jahre.
\item[AS:] Und an dieser Schule?
\item[IP1] Auch 14 Jahre
\item[AS:] Und welche Fächer?
\item[IP1:] Wirtschaftsinformatik und alles Nachfolgende wie Angewandte Informatik, Informationsmanagement und auch die Ausbildungsschwerpunkte Cross Media und Mediendesign. 
\item[AS:] Aus den Unterrichtsfächern erkenne ich schon eine gewisse positive Einstellung zu digitalen Medien. Wenn ich das so global fassen darf, es gibt ja diesen Grundsatz, dass man die Schülerinnen und Schüler versucht in ihren Lebenswelten abzuholen und in diesen Lebenswelten dann im Unterricht begegnen soll. Die Lebenswelt der Schülerinnen und Schüler ist einfach sehr digital mit Mobiltelefon, Streaming und was auch immer. Welche Einstellung hast du grundsätzlich zu digitalen Medien bzw... zu all den neuen technischen Dingen?
\item[IP1:] Ja, ich bin prinzipiell sehr interessiert an neuen Medien, neuen Anwendungen, neuen Geschäftsformen und Geschäftsfelder. Ich muss aber sehr oft auch dann erkennen, dass es schwierig ist damit umzugehen. Ich erkenne auch die Probleme die dadurch entstehen, vor allem bei Jugendlichen, die das teilweise sehr unreflektiert verwenden. Aber auch die Gefahren für die gesamte Gesellschaft oder die Personen die unreflektiert damit arbeiten. Ganz nach dem Motto: Es kostet immer alles etwas. Wir bekommen sehr oft vorgegaukelt, dass es nichts kostet und diese versteckten Kosten, die Weitergabe von Informationen, sehe ich als etwas schwierig.
\item[AS:] Das heißt, ich höre da auch schon eine gewisse Skepsis auch mitschwingen aber eine grundsätzliche positive Haltung aber mit einem sehr großen Bewusstsein oder Fokus wo es sozusagen Nachteile für Schüler unsere geben könnte. Wenn ich kurz nachfragen darf im privaten Bereich welche technischen Devices oder Dienste nutzt du da?
\item[IP1:] Ja also Smartphone, PC, Internetanbindung, Tablets, ich bin zu hause mit WLAN ausgestattet. Aber so die Grenze ist irgendwie Sprachsteuerung beim fernsehen oder die Alexa von Amazon. Da sind für mich die Grenzen überschritten.
\item[AS:] Also bei Sprachsteuerung sind das dann vielleicht doch sensible Informationen?
\item[IP1:] Einfach das Nicht-Wissen wie viel weitergegeben wird und was angehört wird.
\item[AS:] Aber so Online Shopping und Telebanking...?
\item[IP1:] Ja, das mach ich schon.
\item[AS:] Und vielleicht zu deiner persönlichen Ausbildung: War das in deiner Ausbildung damals Thema, was du im Lehramtsstudium damals gemacht hast? Der Umgang mit Medien und Computer grundsätzlich?
\item[IP1:] Nein, dazu ist das schon zu lange her. Da hat es die PCs als PCs geben und das war nicht so verknüpft mit dem alltäglichen.
\item[AS:] Aber eine grundsätzliche Handhabe mit einem PC wurde schon gelehrt, oder?  
\item[IP1:] Nicht wirklich gelehrt worden. Das ist so nebenbei mitgegangen.
\item[AS:] Hast du dann Fort- oder Weiterbildungskurs in diesem Bereich, grundsätzlich jetzt Computereinsatz oder Handling mit Computer besucht?
\item[IP1:] Nicht wirklich aber durch meine Tätigkeit an der Schule und durch die Beratung also autodidaktisch einfach weiter gelernt oder auch immer noch offen für Neuerungen oder wie man das verbessern kann oder wie es neue Szenarien gibt die das verbessern.
\item[AS:] Das heißt es hat in der Hinsicht keine verpflichtende Fort- oder Weiterbildungsseminare gegeben wo du sozusagen dabei sein musstest?
\item[IP1:] Es gibt beispielsweise dieses Safer Internet, das gefördert wird diese Safer-Internet-Days die wir eigentlich bei uns an der Schule nie wirklich gemacht haben.
\item[AS:] okay dann würde ich gleich zum Unterricht kommen. Setzt du digitale Unterrichtsmittel im Unterricht ein?
\item[IP1:] Ja
\item[AS:] Welche sind das und warum? Wenn du vielleicht kurz erzählst wie der Unterrichtsalltag ausschaut.
\item[IP1:] Ich habe in jedem Unterricht eine Unterrichtsvorbereitung digital aufbereitet. Ich verwende eine Lernplattform, das LMS. Ich beziehe mich sehr viel auf Artikeln, die ich als Link her gebe. In den wenigsten fällen Kopien. Auch Kontrollfragen, E-Books, also Unterlagen für die Schüler.
\item[AS:] Und warum?
\item[IP1:] Weil es andere Bücher nicht gibt zu gewissen Ausbildungsschwerpunkte und Unterrichtsgegenständen und auch aufgrund der Aktualität. Weil ich finde, damit kann man es eigentlich besser gestalten, dynamischer gestalten, wenn sie was ändert.
\item[AS:]Und in welchen Phasen des Unterrichts setzt du digitale Unterrichtsmittel ein? Nur zur Sicherung des Lernertrages, Lernkontrollen, zum Einstieg?
\item[IP1:] Durchgehend. Zum Einstieg, Informationsvermittlung, Wiederholung, Kontrolle.
\item[AS:] Gibt es Phasen die sich besser dafür eignen aus deiner Sicht oder die sich weniger gut eignen oder wo du sagst: In dem Bereich setze ich es überhaupt nicht ein, aus irgendwelchen Gründen?
\item[IP1:] Nein, gibt es nicht.
\item[AS:]Auch bei Schularbeiten?
\item[IP1] Ich verwende es auch bei Schularbeiten.
\item[AS:] Und diese Unterrichtsmaterialien, das habe ich vorher schon mitbekommen, die erstellst du selber oder gibt es auch etwas bestehendes, was du verwendest von Verlagen oder von anderen Quellen, wo es fertige Unterrichtsmaterialien gibt oder ist das selbst erstellt?
\item[IP1:] Das meiste ist selbst erstellt. Strukturen und Einteilungen schau ich mir dann von Lehrbüchern ab.
\item[AS:] Okay das wäre im Prinzip schon der erste Fragenblock gewesen ich fasse das nur für mich noch einmal zusammen:
Es gibt eine grundsätzliche positive oder aufgeschlossene Einstellung gegenüber von neuen Technologien, Neuen Medien. Wobei auch ein großer Fokus immer dieses reflektierte Einsetzen ist auch mit Nachteilen oder auch mit Gefahren verbunden ist. Also dass man das immer sehr gut reflektiert und diese neuen Medien eben sorgfältig oder sorgsam auswählt und auch einsetzt. Du verwendest es sehr viel, auch im privaten Umgang wobei es eine Grenze gibt. Aber auch im privaten Umfeld verwendest du neue Technologien und neue Medien. Und im Unterricht ebenso bildest du mehr oder weniger komplett deinen gesamten Unterricht mit digital Unterrichtsmittel ab die auf einer Lernplattform bereitgestellt sind. Stimmt das so ungefähr?
\item[IP1:] Ja.
\item[AS:] Gut dann würde ich gerne zum zweiten Teil kommen. Ich habe im Zuge meiner Recherchen Untersuchungen gefunden, dass 
die Haltung eines Lehrenden gerade im Berufseinstieg sehr stark vom Umfeld im Konferenzzimmer abhängt. Das trifft für unseren Fall nicht ganz zu aber aus deiner Sicht wie würdest du denn die Einstellung des Lehrkörpers grundsätzlich zum Einsatz von digitalen Unterrichtsmittel beschreiben wenn du sozusagen jetzt und das Konferenzzimmer denkst? In dieser schule ist es ein großes Konferenzzimmer. Kannst du da vielleicht etwas erzählen?
\item[IP1:] Ich würde es differenziert sehen. Es gibt einen Teil, die sehr aufgeschlossen sind und die digitale Medien nutzen und einen Teil die, aus welchem Grund auch immer, dem eher skeptisch gegenüber stehen und die auf die vertrauten, althergebrachten Methoden vertrauen und mit diesen arbeiten. Ich glaube auch nicht dass man da noch etwas ausschöpfen kann. Ich glaube nicht, dass man ein Lager von einem Lager ins andere jemanden holen kann. Die, die es nutzen, digitale Inhalte, die verwenden es. Die werden sich nicht zurück beschränken auf andere Inhalte und die anderen wird man auch nicht mehr ins Boot holen können. Ich glaub, das ist schon aus differenziert.
\item[AS:] Wenn du das so pauschal siehst: Gibt es eher eine positive Stimmung in Bezug auf digital Unterrichtsmittel oder eher eine negative?
Fühlst du dich ein bisschen als Außenseiter oder fühlst du dich sozusagen in der Masse wohl?
\item[IP1:] Das kann ich so nicht sagen. Ich weiß nicht ob es mehr als die Hälfte sind. Ist man da ein Außenseiter, wenn man bei einem Drittel dabei ist?
\item[AS:] Und wenn du das vielleicht versuchst das für die Fachgruppe abzuschätzen?
\item[IP1:] Ich glaube nicht, dass ich ein Außenseiter bin. Das passt. Die anderen setzen das auch ein. Verschieden tief halt und verschieden oft.
\item[AS:] Beeinflusste die Haltung des Kollegiums deinen Unterricht in Bezug auf digitalen Medien?
\item[IP1:] Nein, da gibt es keinen Einfluss.
\item[AS:] Und gibt es von der von der Direktion zum Beispiel Initiativen die den Einsatz von digitalen Medien im Unterricht fördern?
\item[IP1:] Es hat immer eine gute Zustimmung gegeben. Das ist immer wertfrei hinterfragt worden. Also da gibts auch keine Hindernisse.
\item[AS:] Und aus seiner Wahrnehmung: könnte sozusagen eine positives Motivieren von der Direktion für den Einsatz förderlich sein oder neutraler oder auch hinderlich sein?
\item[IP1:] Ich glaube, es ist neutral. Es ist schon aus differenziert. Gewinnen kann man nur noch die Junglehrer.
\item[AS:] Und glaubst du das es so Initiativen vom Ministerium oder Landesschulrat, wenn da etwas kommt, ich denke jetzt an eLearning Clusterinitiativen oder solche Initiativen, dass die den Einsatz fördern können?
\item[IP1:] Nein, das glaub ich auch nicht. Ich glaube, dass man nur in der Lehrerfortbildung das aufzeigen kann. Den Standardweg, also das, was bis jetzt immer gemacht worden ist oder eben einen eher digitalen weg. Dass dort am ehesten etwas zu gewinnen wäre.
\item[AS:] Okay das wäre eigentlich auch schon der zweite Fragenblock gewesen. Fällt dir dazu noch was ein in diesem Bereich: Berufliches 
Umfeld oder so?
\item[IP1:] Nein.
\item[AS:] Wenn ihr das kurz zusammenfasse haben sich also deiner Meinung nach die Leute schon mehr oder weniger schon entschieden digitale Unterrichtsmittel einzusetzen oder auch nicht und aus deiner Wahrnehmung nach kann das von außen sehr wenig gesteuert werden. Weder von Direktion noch vom Landesschulrat. Okay danke und dann würde ich gerne zum letzten Fragenblock gehen. Da geht es um die Technik und  Infrastruktur. Und zwar ist ja die Grundvoraussetzung dass man digitale Unterrichtsmittel im Unterricht verwendet, dass eine  entsprechende Infrastruktur vorhanden ist. wenn ihr da konkret nachfrage: Wenn du an deinen Unterricht denkst, findest du alle erforderliche Mittel und technische Infrastruktur funktionstüchtig vor, um den Unterricht so zu abwickeln zu können wie du ihn brauchst und was braucht es dazu eigentlich?
\item[IP1:] Ja ich glaub schon, dazu haben wir ja schlussendlich selbst gesorgt. Es bedarf einer vernünftigen Internetanbindung mit einem vernünftigen Durchsatz. Es bedarf einer guten Netzwerkstruktur. Das heißt in jeder Klasse Netzwerkanbindung und ein ausreichendes WLAN, eine ausreichende Abdeckung mit WLAN weil wir vielleicht in Bezug auf Bring your own device und den Einsatz von Handys und für Abstimmungen ,zum Lösen von Aufgaben in Klassen wo es beispielsweise nicht jeder Schüler einen PC zur Verfügung hat, dass man sich ja so den klassischen Unterricht mit Tafel oder Beamer und dann teilweisen Einsatz über Handys. Da braucht man dann ein WLAN, ein vernünftiges. Ich glaube, dass mit dem ist das dann abgedeckt. PC und Beamer in jeder Klasse, das setze ich jetzt mal voraus.
\item[AS:] Okay das ist in der Schule natürlich vorbildlich gelöst aber es gibt auch Schulen, wo das nicht so der Fall ist. Fehlt dir etwas, um den Unterricht nach deinen Vorstellungen umzusetzen? Was wäre so ein "Nice to have"? Also wenn jetzt Wunschkonzert wäre, was wäre noch nett? bräuchte es noch etwas?
\item[IP1:]Da wüsste ich jetzt nichts. Ich habe jetzt aber auch keine Erfahrung mit Tablet-Klassen, oder so. Wie wäre es, wenn jetzt jeder Schüler ein eigenes Tablet mitbringen würde? Für meinen Unterricht habe ich ja PCs. Mir fällt nichts ein.
\item[AS:] Okay man ich nehme jetzt fast mit ja die Antwort vorweg, aber fühlst du dich sicher im Umgang mit den vorhandenen technischen Equipment? Computer, Beamer, Lautsprecher, Software auf den Computer?
\item[IP1:] Ja.
\item[AS:] Aus deiner Position heraus natürlich klar. verwendest du im Unterricht auch eigenes Equipment? Laptop, Tablet?
\item[IP1:] Ja teilweise. Manche Kabel und Adapter. Im großen und ganzen bin ich ausgestattet von der Schule. Das ist glaube ich jetzt eher die Ausnahme aufgrund des Kustodiates.
\item[AS:] Und warum verwendest du das eigene Equipment? Weil es nicht vorhanden ist oder weil es in zu geringer Stückzahl vorhanden ist oder weil es nicht auffindbar ist? Was sind da so die Gründe?
\item[IP1:] Weil es schlicht und einfach nicht vorhanden ist und der Aufwand der Beschaffung zu  groß ist oder weil es zu wenig Bedarf da gibt für gewisse Adapter.
\item[AS:] und wenn etwas nicht funktioniert, gibt es da die entsprechende funktionieren Supportstrukturen im Haus?
\item[IP1:] Ja, die gibt es.
\item[AS:] Fällt dir noch etwas ein zum Bereich Technik und Infrastruktur, was du vielleicht noch ergänzen könntest?
\item[IP1:] Nein.
\item[AS:] Gut dann fasse ich auch das zusammen. Also wenn ich das richtig verstanden habe, ist die Schule sehr gut ausgestatteten mit allem. in jeder Klasse sind sozusagen die Beamer und Computer und so weiter vorhanden. Also das heißt grundsätzlich würde es daran nicht scheitern wenn man digitale Unterrichtsmittel einsetzen will, kann man das sowohl in den EDV-Räumen als auch in einer entsprechenden Variante auch in 
jedem Klassenraum machen. Die Supportstrukturen sind vorhanden und im Prinzip ist auch das Equipment da.
 Gut dann danke ich für die für das Interview und ich lasse dir die Ergebnisse dann zu kommen.
\end{itemize*} 
}
\section*{Interview 2} 
\texttt{
\begin{itemize*} 
\rightlinenumbers*
\modulolinenumbers[5] 
\linenumbers[0] 
\item[AS:] Gehen wir gleich zum ersten Teil.
Zum einstieg: Es gibt ja diesen pädagogischen Grundsatz, dass man
versucht unsere Schüler dort abzuholen wo sie sich gerade in
ihrer Lebenswelt auch befinden. Und eine
dieser realen Lebenswelten in der sich
unsere Schüler befinden ist natürlich das
Handy und mit allem was dazugehört. Für Schüler ist es ja eigentlich nicht
aus ihrem Alltagsleben wegzudenken. was
glaubst du welche Rolle sollte deiner
Meinung nach das Handy oder das mobile Device in der Schule spielen?
welche Erfahrungen hast du oder welche
Einstellung hast du da dazu?
\item[IP2:] Also bis vor wenigen wenigen Jahren waren das
ausschließlich negative Erfahrungen mit dem Handy. Der Handygebrauch im
Unterricht, der nicht erlaubte, ist doch ein großes Problem. Man muss also die Handys
ständig weg räumen lassen von den
Tischen, die Schüler machen das nicht von
alleine schon in der Pause, man muss sie
dazu auffordern.
dann gibt es immer wieder Schüler die
das Handy verwenden und die  müssen es dann nach vorne bringen zum Lehrertisch. Wir kennen alle diese
Spielchen. Allerdings seit einigen Jahren bereits setze ich das Handy auch im
Unterricht ein. Zwar ganz bewusst. Und zwar aus folgenden praktischen
Überlegungen wenn sich spontan oder
geplant eine Recherche im Internet
anbietet dann haben wir nicht unbedingt
gleich einen EDV-Saal zur Verfügung. Also wir als 
naturwissenschaftliche Lehrer hätten ja den Vorteil, weil wir haben ja hier die EDV-Geräte
zur Verfügung aber der Saal ist auch
nicht immer frei und wie gesagt es kann
da durchaus sich spontan aus dem
Unterrichtsgeschehen etwas ergeben das
wir halt was Aktuelles recherchieren und erteile ich durchaus die Aufforderung dass die Schüler jetzt
das Handy heraus nehmen sollen
und eben diese Suchbegriffe eingeben
sollen und vielleicht etwas herausschreiben, etwas vergleichen oder eine Frage, die ich an
die Tafel schreibe diesbezüglich zu
beantworten. Also so läuft das ab uns am
Anfang war das ganz interessant weil
ich eher ein Lehrer bin, der wirklich versucht
gegen die Handys zu arbeiten und dann plötzlich
wenn man in einer Klasse das erste mal sagt, dass
die Schüler die Handys herausnehmen sollen, dann schauen sie dich mit großen Augen an.
\item[AS:] Und wenn ich zurückkommen darf zu den Nachteilen und den Problemen. Was sind das hauptsächlich für
Probleme ?
\item[IP2:] Dass die Schüler abgelenkt sind. Dass sie der Unterricht langweilt, dass sie an 
Konzentrationsproblemen leiden. Oder wo sie halt gerade nichts tut, wo der Lehrer etwas erklärt, wo sie nicht schreiben müssen, wenn sie kein Arbeitsblatt bearbeiten müssen, dass sie  dann nach dem Drang nachgeben und aufs Handy halt schauen.
\item[AS:] Es ist
hauptsächlich die Ablenkung?
\item[IP2:] Ja, es ist die Ablenkung vom Unterricht und die ist massiv und
das ist nach wie vor so. Aber ein bisschen kann man halt eine Tugend daraus machen.
\item[AS:] So die grundsätzliche
Einstellung zu digitalen Medien: welche
hast du da? Ist das für dich ein positiv
besetztes Thema, ein negativ besetztes
Thema oder wie stehst du zu digitalen
Medien grundsätzlich in der schule. Also jetzt auch abgesehen vom Handy
\item[IP2:] Also grundsätzlich positiv.
\item[AS:] Das heißt, du setzt auch im Unterricht digitale medien- ein?
\item[IP2:] Ja.
\item[AS:]  Und im
privaten Bereich? Was nutzt du da eigentlich für Technologien?
\item[IP2:]  Also das übliche wie
alle anderen Haushalte auch. Also ihre PCs, Laptops, das Internet für Recherche.
\item[AS:] Auch online-Shopping?
\item[IP2:] Durchaus, ja.
\item[AS:] Telebanking?
\item[IP2:] Telebanking nicht, aber Online-Shopping sehr stark.
\item[AS:] Und auch, dass man sagt
im Unterschied zu dem linearen Medium wie fernsehen oder so
dass man sagt im Internet die 
Nachrichten anschaut, dann an wenn ich es will, oder Internet Video Streams auch?
\item[IP2:] Wenig. 
\item[AS:] Wenn ich da noch ein bisschen nachfragen darf: Es sind ja schon einige Jahre her, wie du die Ausbildung gemacht hast. Inwieweit waren damals in der Ausbildung auf der Uni elektronische Medien schon Thema?
\item[IP2:] Also als ich Diplomarbeit 
geschrieben habe, hatten wir kleine
Rechner mit so einem winzigen Bildschirm, alles schwarz-weiß. Wir mussten also Befehle
einprogrammieren. Ich habe eine Diplomarbeit gemacht, wo ich statistische Auswertungen verwendet habe mit SPSS, dieses Statistik-Programm. Eine alte Version, wo es so ein großes, dickes Buch gegeben hat. Da hat man sich diese Befehle herausgesucht und diese dann händisch eingetippt. für mich hat
gegen ende des Studiums das eine ganz
wichtige Rolle gespielt, zu Beginn des Studiums eigentlich überhaupt nicht. Das war so gut wie nicht vorhanden.
\item[AS:] Also  eher mehr als Werkzeug wie als Unterrichtsmittel, oder?
\item[IP2:] Also dann als Lehrer am
Anfang als Unterrichtsmittel überhaupt nicht. Ich habe diese Overhead-Folien verwendet. Ich war ein starker Befürworter. Das war auch mein
Stil und das bietet sich in meinem Fach auch an. Wir habe auch viele Atlanten mit Bildern. 
Die Darstellung  von Strukturen sind ja bei uns ganz wichtig und das ging damals mit
dem Overhead am Besten.
\item[AS:] Und andere Medien?
\item[IP2:] Gibt es auch. Zum Beispiel vom Klett-Verlag haben wir einige CD-Roms, die zwar nicht mehr die Neuesten sind, aber immer noch funktionieren und die setze ich
persönlich auch sehr gerne ein.
\item[AS:] Hast du auch Fort- oder Weiterbildungskurse
im Bereich Computereinsatz besucht?
\item[IP2:] Ich hab diese Laptopausbildung gemacht, sonst aber nichts
\item[AS:] War das freiwillig
oder verpflichtend?
\item[IP2:] Naja, man wurde gefragt und hat es dann gemacht.
 \item[AS:] Wenn du jetzt eine typische
Unterrichtsstunde von dir denkst und du
setzt digitalen Unterrichtsmittel ein, 
warum setzt du die dann ein?
\item[IP2:] Der Anschaulichkeit wegen. Also um die Anschaulichkeit zu erhöhen und um auch
mehr Aktualität herzustellen, um die Datenbeschaffung zu erleichtern.
\item[AS:]   Und
in welchen Phasen des unterrichts? Ist es
eher für zum einstieg zum Thema oder  zum
Wiederholen oder zum Festigen, zur Lernkontrolle?
\item[IP2:] Eher 
zum Schluss. Es kommt darauf an: 
Handy wird nur zum Schluss und meine
geliebten CD-Roms zu Beginn als
Unterstützung des  Vortrages, des
erklärenden Vortrages.
\item[AS:] Und so ein repräsentatives
Beispiel wäre zum Beispiel wenn du dabei
 an eine bestimmte Situation denkst?
 \item[IP2:]   Wir machen jetzt in der in der nächsten
Stunde in einer zweiten Tourismusklasse
da haben wir jetzt Reisekrankheiten
durchgekommen man eine Reisekrankheit
nehmen wir jetzt exemplarisch heraus, das ist die Malaria, und da haben wir eine Animation, eine Trickfilmanimation
zur Malaria über den gesamten
Lebenszyklus von der Mücke und des Erregers beschreibt und so weiter und so fort. Und da wird das so erklärt und mit dem beginnt man dann.
\item[AS:]  Erstellst du
deine digitalen Unterrichtsmaterialien selbst oder verwendest du eher
bestehende wie zum Beispiel CDs, wie du auch
anklingen hast lassen oder vom LMS, vielleicht
gibt es da ja naturwissenschaftlich Unterlagen, oder von Verlagen?
\item[IP2:] Also alles was mit Bildern zu tun hat wird eher was verwendet, was gekauft wird. Hingegen eine Powerpoint-Präsentation mache ich mir selbst. 
\item[AS:] Gibt es eine spezielle Situation, wenn du an deinen Unterricht denkst, wo da ganz bestimmt keine digitalen Unterrichtsmitteln einsetzen würdest?
\item[IP2:] Grundsätzlich eigentlich nicht da
müsste ich mich mehr einarbeiten, aber ich hätte auch keine Scheu über elektronischen Methoden abzuprüfen. Aber da bin ich zu wenig eingearbeitet. Die Dinge, die wir mal gelernt haben, mit den Multiple-Choice-Fragen ist zu unflexibel für die Fragestellung. Ich brauch da auch Bilder und der Muliple-Choice-Test ist da zu einseitig zum Prüfen. Aber da hab ich zu wenig Erfahrung.
\item[AS:] Aber grundsätzlich könntest du dir
vorstellen dass du in jeden Phasen, wenn passend, wenn möglich, wenn vorbereitet, ...
\item[IP2:] Natürlich, alles was es erleichtert und die Anschaulichkeit erhöht und auflockert, das ist ja auch das was Filme betrifft.
\item[AS:] Das war eigentlich schon der erste Block. Im zweiten Block gehts um das berufliche Umfeld, um die Schule. Ich weiß nicht, ob du das kennst. Es ist eine Studie aus dem Jahr 1978, die heißt, die "`Konstanzter Wanne"'. Da geht es darum, wie  Berufseinsteiger, 
Lehrer Berufseinsteiger, innerhalb
kürzester zeit mit einem Praxisschock
konfrontiert sind und sich ganz schnell
im Lehrerkollegium sozialisieren
sozusagen so arbeiten wie es im
Lehrerkollegium ist und da gibt es
ganz interessante Studie dazu. Das
hat jetzt nichts mit Medien zu tun, sondern
einfach nur der unterschied zwischen
meiner Ausbildung und dann dem
Wechsel in die Praxis und was sich damit
verändert. Und vor dem Hintergrund dieses Einflusses des
Lehrerkollegiums, wäre die Frage wie
glaubst dass die Einstellung des
Lehrerkollegiums so grundsätzlich
gegenüber dem Einsatz von digitalen
Unterrichtsmittel da im Kollegium? Wie würdest du das Beurteilen, aus deiner Wahrnehmung, aus deiner Sicht?
\item[IP2:] aus meiner Sicht ist es schwierig zu sagen, denn man das sieht ja selten den Unterricht von Kollegen. Ich kann es jetzt
nur von Fachkollegen beurteilen, die diese Medien durchaus auch einsetzen. Also
die andere Kollegin aus Biologie und
Ernährungslehre,  die setzen durchaus auch ein. Teilweise sogar noch verstärkt, viel mehr als ich.
\item[AS:] Also so eine Grundhaltung, 
einer Grundstimmung kann man jetzt nicht
irgendwie heraus spüren?  Also
eine grundsätzlich positive oder so
grundsätzlich negative Stimmung?
\item[IP2:]  Ich denke schon eher positiv, würde ich meinen.
 \item[AS:] Und das ist jetzt wahrscheinlich auch eher 
fachgruppenspezifisch oder zumindest hat man da mehr Einblick als in anderen
\item[IP2:] Ja.
\item[AS:]Und glaubst du, dass die Haltung des
Kollegiums deinen Unterricht in Bezug
auf den Einsatz von digitalen Medien
beeinflusst?
\item[IP2:]  Beeinflusst 
durchaus schon aber doch relativ wenig.
\item[AS:] Du hast vorher gesagt, dass die unmittelbaren Fachkollegen
setzen vielleicht sogar mehr ein.
würdest du da jetzt zu empfinden: Jetzt muss ich auch mehr einsetzen, oder?
\item[IP2:]  Nein, 
da lasse ich mich nicht irgendwie beeinflussen. Ich versuche das immer das für mich optimal
abzustimmen.
\item[AS:]  Und, hypothetisch
umgekehrt, wenn es wenn es so wäre dass
die beiden anderen gar nix
einsetzen würden, 
würdest du dich auch nicht beeinflussen
lassen?
\item[IP2:] Nein.
\item[AS:]  Gibt's von der Direktion her
irgendwelche Initiativen oder so dass
man digitale Medien im Unterricht mehr einsetzt oder fördert?
\item[IP2:] Also im Moment gibt es keine Kampagne, die wär mir jetzt nicht bewusst.
Empfehlung hat es immer wieder gegeben, für LMS zum Beispiel.
und auch dieses Bildungsmedien-TV jetzt. Auch eine Initiative, die vom Heinz ausgegangen ist. Das ist eine ganz tolle Sache.
\item[AS:] Und 
wenn jetzt die
Direktion zum Beispiel sagen würde, es ist wichtig dass du jetzt in jedem fach
LMS einsetzt. Inwieweit würde
dich das jetzt beeinflussen, beunruhigen oder nicht
beunruhigen oder bestätigen oder wie wäre das?
\item[IP2:] Also ich könnte das durchaus einsetzen
was ich im Moment nicht tue. 
Es müssten nur zu dem passen was ich vermitteln möchte. Ich verwende LMS, da wo es notwendig ist. Bei ein paar Dingen gehört es dazu. Aber so im normalen Unterricht, zur Stoffvermittlung setze ich LMS nicht ein.
\item[AS:] Ist vermutlich auch schwieriger wenn man keinen EDV-Raum hat. Gibt es vom Landeschulrat oder vom Ministerium Initiativen, die dir bewusst sind oder dich unmittelbar betreffen? 
\item[IP2:] Außer LMS fällt mir nichts ein.
\item[AS:] Gut, danke, dann kommen wir auch schon zum dritten Block. Es gibt, und da bin ich durchaus sehr kritisch, auch wenn ich sehr viele digitale Unterrichtsmittel einsetze, das Problem sehr oft, dass 
wenn jetzt jedes jeder Schüler irgendein
mobiles Device oder Tablet oder Notebook
oder was auch immer, das es ja immer
wieder Situationen gibt, und die sehr
häufig, dass irgendetwas nicht
funktioniert und in Wahrheit gehen dann
im Endeffekt relativ viel wertvolle
Unterrichtszeit drauf um technische
Probleme zu beheben. Das ist ja auch ein
wirklich großer Nachteil oder auch ein
Argument, das man sehr oft hört. Wenn du
jetzt an deine Schule denkst und an deine alltäglichen Umgang mit Technik
und Medien, ist da für dich die
Infrastruktur in Ordnung? Funktioniert
das aus deiner Wahrnehmung heraus?
\item[IP2:] Also bei diesen Dingen, es liegt glaube ich in der menschlichen Natur, dass man Einzelfälle über
die man sich ärgert wenn etwas nicht
funktioniert eher hervorhebt und die Routine, wo eh alles klappt, das ist ja der Normalfall.
Ärgerlich sind Dinge wie, wo der Schulserver nicht
funktioniert und man kann nicht einsteigen und auf den Kommunikationsordner zugreifen. In letzter Zeit ist das seltener. In der Vergangenheit war das aus 
meiner Wahrnehmung öfter, gehäuft aber
das ist rein individuelle Wahrnehmung.
Aber wahrscheinlich überspitzt dadurch
dass es jetzt nicht geht und ich das jetzt
unbedingt brauche. Und diesen Ärger merkt man sich dann.
aber ich habe meine Dateien, die ich verwende auf einem Stick. Das hab ich mir angewöhnt
den Stick einfach automatisch
anzustecken, denn der funktioniert
immer und dann so kann man dem
Kommunikationsordner ausweichen. Und dass einmal der Beamer nicht funktioniert, diese Kabelverbindung, das passiert halt.
\item[AS:] Und hast du das Gefühl, dass dir
irgendetwas fehlt um den Unterricht nach
deinen Vorstellungen abzuwickeln oder
umzusetzen?
\item[IP2:] Aus der EDV-Sicht, nein.
\item[AS:]Auch in den Klassenräumen?
\item[IP2:]Nein.
\item[AS:] Wenn wir dieses Gedankenexperiment "`ich dürfte
mir wünschen was ich wollte"' spielen,  was wäre das technisch gesehen?
Würde es da etwas geben? Das wäre für meinen Unterricht notwendig, damit ich ihn genauso machen könnte, wie ich gerne möchte?
\item[IP2:] Also Handymikroskope haben wir jetzt angeschafft bekommen. Das sind so kleine Aufsätze auf den Handys, die man im Unterricht einsetzen kann.das ist eine ganz wunderbare Sache. Das ist so für die letzten 15 Minuten einer Stunde. Das ist sehr spannend.
\item[AS:] So 
grundsätzlich fühlst du dich sicher im Umgang mit dem vorhandenn technische Equipment wie, Beamer, Lautsprecher
Computer, mit den Programmen?
\item[IP2:] Also was ich brauche das kann ich. Ab und zu, wenn dann mal ein Beamer nicht funktioniert, das ist mir nicht immer ganz
geläufig dass hat dass man das Bild auf
den Beamer bekommt oder das mit dem Bildschirm teilen. Das hat mir witzigenweise ein Schüler erklärt aber ich
hab nicht mitgeschrieben. Aber dann hat man immer wieder Schüler in der Klasse, die da weiterhelfen können.
\item[AS:] Ich hab das mit ein bisschen einem Hintergedanken
angesprochen, weil auch in der Literatur
kommt vor dass manche Lehrpersonen technische oder
digitale Medien nicht einsetzen weil sie
selber ein bisschen unsicher fühlen.
\item[IP2:] Nein, unsicher fühle ich mich nicht. 
\item[AS:] Auch vor dem Hintergrund auch diese frage noch, ich habe jetzt rausgehört, du fragst einfach Schüler, die dir dann helfen und du hast auch kein Problem damit. Aber das ist für manche auch eine Hemmschwelle.
\item[IP2:] Nein, man kann nicht alles wissen. Und bei diesen Dingen ist es durchaus so, dass einem die 15, 16-jährigen voran sind.
\item[AS:] Verwendest du im Unterricht auch eigenes Equipment? Ein eigenes Notebook
oder eigenes Tablet?
\item[IP2:] Überhaupt nicht.
\item[AS:] Vielleicht noch so kleine frage
zum Schluss: Wenn einmal was nicht
funktioniert gibt es die passende
Unterstützungsstrukturen im Haus und
funktionieren die auch?
\item[IP2:] Die gibts natürlich, diese Fehlermeldungsverknüpfung am Desktop. Ja das funktioniert.
\item[AS:] Vielen Dank für das Interview.
\end{itemize*} 
}
\newpage
\section*{Interview 3} 
\texttt{
\begin{itemize*} 
\rightlinenumbers*
\modulolinenumbers[5] 
\linenumbers[0] 
\item[AS:] Es gibt einen Grundsatz dass man sagt
man versucht die Schüler dort abzuholen
wo sie sich in ihrer Lebenswelt
befinden und das Handy ist ja sozusagen
ein zentraler Bestandteil der Lebenswelt
unserer Schülerinnen und Schüler.
Welche Rolle sollte deiner Meinung nach
das Handy in der Schule spielen? 
\item[IP3:] Also ich glaube das ist eine
Altersfrage weil ich glaube wenn die
Schüler Unterstufe sind und so mit zehn, elf
in die unterstufe kommen dann können sie
teilweise noch nicht umgehen mit dem
Handy oder müssen das auch lernen und da
glaube ich muss die Schule auch Regeln
aufstellen einfach, weil ich glaube sie
sind einfach noch überfordert das eben
nicht ständig an dem Handy sind, in den
Pausen nicht irgendwelche
Spiele spielen, etc.
Ich glaube man nicht dass es Sinn macht
das wegsperren lässt vor allem nicht
wenn sie älter werden weil ich dann
glaube, oder ich auch sehe, das sie es wirklich
verwenden. Also sie verwenden es auch zum Wörter suchen, zum Beispiel.
Sie
verwenden es ja auch für den Unterricht
oder grad im Sprachunterricht als
Aufnahmegerät zum Beispiel. Aber ich
glaube schon dass es prinzipiell Regeln
braucht vor allem in der Unterstufe.
\item[AS:]  Und
jetzt konkret in der Situation bei dir im Unterricht in der Oberstufe? wie siehst
du den Umgang oder den Einsatz vom Handy?
\item[IP3:] Also wenn ich weiß dass ich es auf keinen
Fall brauchen in der Stunde und es mich
stört dann bin ich schon dafür dass ich
sage wir legen sie ist auf die Seite, alle
legen es hinten am Tisch und sonst
wenn wir es brauchten im Unterricht dann
können sie es verwenden und auch wenn
sie eigenständig arbeiten also wenn ich
jetzt nicht vorne stehe und frontal unterrichte sondern sie
Arbeitsaufträge haben dann stelle ich es
ihnen eigentlich frei weil ich denke
auch dass sie lernen müssen wenn sie das
Handy haben, ja wenn ich zehn Minuten
dann trotzdem mich ablenken lasse
und das kann ich nicht immer überprüfen
ob das jetzt zu tun hat mit dem Unterricht
oder ob das zu tun hat mit irgendwelchen privaten Dingen.
Das müssen sie auch
lernen wie man damit umgeht.
\item[AS:]  Das heißt
ich höre auch heraus, dass das Handy auch in
eine Möglichkeit der Ablenkung ist?
\item[IP3:]Natürlich
ja.
Ich glaube auch je jünger sie sind desto
mehr Unterstützung brauchen sie
da.
\item[AS:] Setzt du auch andere Devices in deinem Unterricht ein?
\item[IP3:]Unsere Schüler
haben ab der dritten Klasse Notebooks.
\item[AS:]Besteht die Notwendigkeit, dass sie beides benutzen? Also Notebook und Handy? Oder in
welchen Unterrichtssituationen wird
welches Medium dann eingesetzt?
\item[IP3:] Also meistens ist es einfach so dass sie in der ersten und zweiten klasse kein Notebook haben
und da rennt dann halt alles über das
Handy weil sie können ja auch
interaktive Übungen zu greifen am Handy
sie können sogar Hörübungen am Handy
machen. Wenn sie das Notebook haben denn
wie ich eigentlich immer davon aus dass
sie das Notebook verwenden auch für Textaufnahmen zum Beispiel. Manche verwenden
aber dann trotzdem das Handy
also ich glaube ich denke eher Richtung
Notebook weil für mich das Handy trotzdem
noch nicht so...
also für mich ist das wenn man eine eine
Audiodatei hochlädt ist für mich
eigentlich klar, das macht man am Notebook. Aber ich glaub, das ist eine Altersgeschichte. Für sie ist es kein Problem
all diese schritte, wo ich ans Notebook denke,
am Handy auch zu machen.
\item[AS:]  Also ich erkenne
an deinen Ausführung eine gewisse Affinität
zu digitalen Unterrichtsmitteln und den entsprechenden digitalen Devices. setzt auch solche Devices im privaten Bereich ein? Notebook
oder oder Online-Dienste?
\item[IP3:] Ja also alles!
Online-Shopping sowieso, Nachrichten, fernsehen, Musik hören, Reisen buchen
Verkehrsmittel suchen, recherchieren
eigentlich alles heutzutage.
\item[AS:] Und hast du diesen Umgang mit dem
Computer oder mit dem Notebook in
der Ausbildung gelernt?
\item[IP3:] Nein, also
überhaupt nicht.
Also ich habe die letzten Präsentationen
auf der Uni mit Overheadprojektor gemacht. Und
ich habe eigentlich 
ich habe sehr wenig Umgang mit dem PC
gelernt weder in der Schule, im Gymnasium,
noch im Studium.
Ich habe sehr viel eigentlich dann "on-the-job" gelernt eigentlich. In der HAK Eisenstadt mit den Notebookklassen und
im LMS-Team, einfach auch von den Kollegen
muss man sagen.
\item[AS:] Das heißt Fort- und
Weiterbildungskurse während deines
aktiven Dienstes als Lehrer hast du auch
besucht in dem Bereich Mediennutzung, Notebookumgang
\item[IP3:] Eigentlich nicht. Wir haben am Anfang, als ich vor 16 Jahren angefangen
habe, hatten wir diese tekomp, glaube ich
habe diese Bücher geheißen, dieses Selbstlernbücher, da habe ich mich schon damit
beschäftigt und habe einen Sommer lang
auch ganz intensiv, ich glaube das hat
noch geheißen, eLisa-Sommerschule oder so.
Wie mach ich Powerpoint? Wie mache ich
Bildbearbeitung? Also ich habe da schon, gerade am Anfang vom unterrichten, sehr viel
Selbststudium gemacht. 
\item[AS:] Wenn du jetzt an
eine eine normale Unterrichtsstunde
in deiner schule denkst wo die
Schüler Notebooks haben, in welcher Phase vom
Unterricht setzt du das Notebook ein oder könntest du so ein Beispiel
geben wie sie eine typische
Unterrichtsstunde ausschaut, wo die
Schüler Notebooks oder Handys einsetzen?
\item[IP3:] Also ich würde da unterscheiden zwischen
erster zweiter dritter vierter fünfter (Jahrgang).
weil dritter, vierter und fünfter haben die
Kinder bei uns eben alle Notebooks und da
würde ich jetzt gar kein Unterschied
machen
da ist das Notebook einfach immer da
weil die Lernplattform immer da ist. Weil alle meine Materialien, 
Arbeitsaufträge eigentlich am Notebook
sind und selbst wenn ich was vortrage, eine Präsentation, 
dann 
haben die Schüler auch die Präsentation
oder haben das E-Buch auch vor sich, oder
schreiben mit oder schauen was nach. Also
da ist es irgendwie immer präsent. In der
ersten und zweiten Klasse würde ich das nicht so
sehen. da wird das Handy irgendwie gesagt
ganz konkret also meistens, für das Aufnehmen muss ich sagen. Dass ich
sie raus schicke oder in Gruppenarbeiten
und sie müssen zurückkommen mit einem fertigen Produkt. Oder sie verwenden es als
Nachschlagewerk, Informationen suchen, wenn sie ein Plakat erstellen oder Vokabel nachschlagen.
\item[AS:] Das heißt in jeden Unterrichtsphase?
\item[IP3:] Ja.
\item[AS:]Die
Unterrichtsmaterialien, verwendest du
bestehende Unterrichtsmaterialien von Verlagen oder vom LMS oder von
sonstigen Quellen oder erstellst du die selber?
\item[IP3:] Also ich erstelle viel selbst auf
LMS habe ich verwende auch Materialien
auf LMS die andere Leute erstellt haben.
\item[AS:] Okay gibt es manchmal Situationen oder
gründe dass man jetzt sagt muss man
setzt digitale Unterrichtsmittel nicht
ein oder in welchen Situationen könnte
das deiner Meinung der Fall sein oder 
warst du irgendwie noch nie in dieser
Situation? 
\item[IP3:] Oja, also ich denke wenn man
wirklich will dass sie sich auf..., also es ist schon eine Ablenkung glaube ich wenn
du das Notebook hast und wenn ich will
dass sie sich wirklich aufeinander
konzentrieren. Ich mag es zum Beispiel
nicht wenn wir uns vorbereiten für der
Diskussionsrunde
und sie kommen dann mit dem Notebook, weil
dort haben sie ihre Notizen gemacht. Weil
ich dann das Gefühl habe, denn wenn dazwischen ein
Chatfenster aufgeht oder Facebook
meldet sich, oder was auch immer sie
alles verwenden, ist das eine
Ablenkung. Wenn ich wirklich will dass
sie da sind und sich entweder
aufeinander konzentrieren oder auch auf
mich konzentrieren. ich habe das beim
Schriftverkehr auf bemerkt da mache ich
sehr viel frontal werde auch ein gutes
Buch lesen ein gutes Skript und da geht es
einfach darum Dinge schnell zu
vermitteln und sobald das Notebook dann
offen ist sie es nicht explizit
brauchen sind sie einfach wo anders. Das ist ja bei Lehrerfortbildungen nicht anders.
\item[AS:] Genauso diese Ablenkungen sind ja auch
ein zentraler Punkt immer wieder.
\item[IP3:] Also
ich glaube es muss schon Phasen geben wo
man sagt: Handys weg, oder Notebook zu, so zu sagen.
\item[AS:] Okay danke das waren eigentlich
schon die Gedanken zu dem ersten Teil.
Und zweiten Teil geht es um das
berufliche Umfeld auch um die Situation
in der Schule und das Lehrerkollegium.
es gibt ja eine Studie da geht es um den
Praxisschock jungen Lehrerinnen und
Lehrer die in die Schule kommen, sozusagen mit einer gewissen
Erwartungshaltung, in den Berufsalltag
stürzen, und dann einen sogenannten
Praxisschock erhalten, dh es gibt irgendwie
einen eindeutigen Zusammenhang wie das
Lehrerkollegium auf einen selber wirkt
also die gruppendynamischen Effekte die
da stattfinden.
Glaubst du dass das Lehrerkollegium oder
auch auf dich einen Einfluss hat ob digitale Unterrichtsmittel eingesetzt
werden oder nicht? Gibt es da
irgendwie eine Erwartungshaltung
einen druck einzusetzen oder im Gegenteil
oder auch nicht einzusetzen? Oder wie wie
nimmst du das war in deinem Kollegium?
\item[IP3:] Also ich glaube schon dass wenn man in eine
Schule kommt und dort werden Medien
eingesetzt oder teilweise wird es auch
eingefordert von Direktoren, das kenne ich auch, dass man dann einen gewissen
Druck hat und mitschwimmt. Aber ich
glaube das funktioniert vor allem bei
jüngeren Lehrer. Also ich glaube jetzt nicht,
dass ältere Lehrer die da
irgendwie schon gefestigt sind in ihrer Rolle,
wenn sie nicht von etwas überzeugt sind
das auch wirklich einsetzen sondern das einfach
aussitzen.
Bei uns im Lehrkörper, wir haben einen 
ziemlich langen weg mit digitalen Medien
Lernplattformeinsatz etc,
das ist eigentlich so
normal mittlerweile sage ich jetzt
obwohl ich nicht weiß wie weit sie das einsetzen, wie
intensiv.
\item[AS:] Das heißt du würdest die Einstellung des Lehrkörpers im
allgemeinen eher positiv gegenüber dem
Einsatz von digitalen Medien einschätzen
aus deiner Wahrnehmung heraus?
\item[IP3:] Es ist zumindest
nicht offensiv dass irgendjemand sagt: 
das ist was schlechtes, ja, was früher
vielleicht mehr noch diskutiert wurde
brauchen die  Notebooks oder: "Ihre die
mit den Notebooks in der Gegend herum rennt, von Klasse zu Klasse"
das ist aber auch, weil so Dinge wie digitales Klassenbuch nun
einfach in den letzten Jahren, ...
es gibt keine Lehrer mehr, die sagen: "den
Computer möchte ich gar nicht
einschalten im Klassenzimmer"
\item[AS:] weil es eine dienstliche Notwendigkeit ist...
\item[IP3:] Es ist ganz normal geworden.
\item[AS:]Glaubst du, dass es Unterschiede oder fachgruppenspezifische
Unterschiede aus deiner Wahrnehmung gibt?
\item[IP3:] Ja das glaube ich schon, Bin mir aber nicht
sicher ob es nur am Fach liegt oder immer
auch an der Schule und dem Fach, sozusagen. Aber ich denke, dass Sprachlehrer,
ich sehe das auch immer wieder bei Seminaren, sind
prinzipiell immer sehr offen all den
neuen Dinge. Gerade in Englisch, weil da
gibt es auch so viel. Sage ich jetzt mal.
An Angeboten oder Ideen wie man
irgendwelche neuen Werkzeuge auch
einsetzen kann.
ich sehe oft Mathematiklehrer irgendwie
ein bisschen skeptisch.
Aber das betrifft auch wieder nur jetzt
die Schule. Das ist eine sehr subjektive
Wahrnehmung! Weil ich kenne außerhalb der Schule
Mathematiklehrer die sehr viel einsetzen. Und Nebenfächer ,... Also die Wirtschaftspädagogen bei uns, die sind auch sehr, also wie soll ich jetzt sagen, innovativ oder gehen
diese schritte mit, aber so Nebenfächer
wie Geographie, Biologie, Religion,  sag ich jetzt einmmal. Also
da bei uns in der Schule hat sich
wirklich fast nichts getan.
Also wird sich schon etwas getan haben, aber
nicht die Bereich der digitalen Medien.
\item[AS:] Und beeinflusst die Haltung des
Kollegiums deinen Unterricht in bezug
auf den Einsatz von digitalen Medien?
\item[IP3:] Nein, eigentlich nicht. Weil ich mir denke grad
jetzt im LMS-Bereich oder was es mit E-Büchern und so mache,  also das ist jetzt sicher
nicht, was die anderen Kollegen unbedingt
machen nur weil ich es mache,  oder was ich
jetzt aufhöre zu machen weil sie es nicht
machen.
Also da es halt jeder Lehrer irgendwie
doch so ein Einzelkünstler. Sowie in
vielen anderen Bereichen.
\item[AS:] Alleinunterhalter. Gibt es vor der
Direktion irgendwelche Initiativen, dass man
digitale Medien im Unterricht einsetzen
soll oder dass die das fördert?
\item[IP3:] Eigentlich nicht. Wenn sind das so Dinge
wie wenn halt vom Ministerium
irgendetwas kommt ja es sind,  schule 4.0
oder bei de Clusterschulen haben wir es gehabt, eine Unterrichtseinheit für jeden
Schüler oder so einem Jahr.
Aber ich erlebt das als sehr
oberflächlich. Also man ist froh wenn
irgendjemand sagt: "Ja das haben wir
gemacht". Wir hakerln das irgendwo ab. Und das war es.
\item[AS:] Also so eine Initiative vom Direktion oder Landesschulrat oder Ministerium, höre ich jetzt irgendwie
raus, trägt nicht unbedingt dazu bei das
man sie einsetzt?
\item[IP3:] Ich glaube es hängt dann
trotzdem an der Führungsperson dann.
Weil ich glaube dass gerade der Landesschulrat für Burgenland zum Beispiel
sehr initiativ ist und es auch überhaupt
schafft das Direktoren dann sagen: "Ihr
müsst das und das machen". Aber es kommt
darauf an wie man das auch selber lebt
und schätzt auch, sage ich jetzt einmal. Und ich glaube dass es in
der BHS sind in der AHS und einiges
schwieriger ist Lehrer und Lehrerinnen
davon zu überzeugen digitale Medien
einzusetzen als jetzt zum Beispiel eine Neuen
Mittelschulen.
\item[AS:] Weil?
\item[IP3] Weil ich erstens glaub,
dass die durch die IT-Betreuer im Burgenland einfach gut betreut sind und dass die Wege
einfacher sind. Also wenn man möchte dass
Lehrer und Lehrerinnen etwas neues
machen sowie Coding und Robotik zum
Beispiel, dann wird das irgendwie mehr
oder weniger verordnet sage ich jetzt
mal und dann geht das gemacht und da
kann ja das Land, das sehr initiativ ist,
auch hat zugriff auf die Lehrer. während
das Land  auf die Bundeslehrer, ...
das ist ein bisschen komplizierter.
\item[AS:]  Okay
dann kommen wir schon zum dritten Teilbereich. es geht um die Technik und
Infrastruktur. Die Erfahrung durch manche
Kolleginnen und Kollegen haben, dass sie
sozusagen beim Einsatz von digitalen Unterrichtsrmitteln, egal in welcher
Form auch immer, es immer wieder
technische Probleme geben kann und
dadurch wertvolle Unterrichtszeit
verloren geht. Wie wird dein Unterricht durch die technische Infrastruktur beeinflusst? wenn du
einen typischen Unterricht von dir
denkst, findest du die erforderlicher Infrastruktur vor und funktioniert diese dann  auch zuverlässig?
\item[IP3:] also eigentlich ja, muss ich sagen. Es
kann schon mal vorkommen dass
Lautsprecher nicht funktionieren, etc. Aber
es stresst mich auch nicht man muss
ich sagen. ich kann mir schon vorstellen, wenn man
neu als Lehrer anfangt, das war bei mir
auch sicher so, dass mich die Technik
sehr gestresst hat. Also ob jetzt
wirklich dieses Notebook geht und ob man
das anschließen kann und das war ja auch
nicht so, dass ich das MacBook hatte das
ging alles irgendwie, sondern das hat
schon gestresst.
wir hatten früher Internetausfälle das
haben nicht mehr. Ich muss sagen wir haben
eine stabile Internetverbindung,
wir haben die Beamer und so, das
passt eigentlich. ja also bei uns rennt
das eigentlich gut.
\item[AS:] das heißt fehlt dir irgendetwas um deine
Unterricht nach deinen Vorstellungen
umzusetzen in der schule oder hast du im Prinzip alles, was du brauchst?
\item[IP3:] vielleicht daneben so Arbeitsplätze für die Kinder
wenn sie halt auf PCs arbeiten sollen, auch in Nicht-Notebookklassen.
Also ich finde, von dem her was vorhanden
ist das rennt eigentlich sehr gut.
\item[AS:] Fühlst du dich
sicher im Umgang mit dem vorhandenen
technischen Equipment? Wenn du jetzt die
Lautsprecher, Beamer, Software uns weiter
denkst?
\item[IP3:]  ja eigentlich schon.  also wenn
dann was nicht funktioniert dann ist nicht
unbedingt so dass ich es jetzt lösen
kann immer, 
aber ich weiß ungefähr was es sein könnte
und ich kann fragen.
\item[AS:] Und du bist so sicher im Umgang dass
du keine Hemmschwelle hast, da etwas einzusetzen?
\item[IP3:] nein. Aber ich bin mir auch bewusst, 
ich mache das jetzt schon ein bisschen
länger, dass einfach was schief gehen kann
oder dass  man halt da nicht das
machen kann was man machen wollte aber
das ist auch kein Problem
\item[AS:] ja und wenn
ich noch mal auf den Eingangsstatement
zurück gehen darf, also glaubst du schon
dass das für junge Kolleginnen und
Kollegen
doch eine eine schwelle sein könnte wenn
sie eben uns unsicher sind im Umgang mit
den Geräten oder mit der Infrastruktur?
\item[IP3:]ja ich glaube schon vor allem weil was ich
halt merke, sie unterschätzen es, also bei
den Studenten merk ich das manchmal. man
muss sich halt doch mit jedem
Klassenraum irgendwie vertraut machen
oder überall läuft es halt anders. also wenn ich jetzt in eine
klasse kommen da ist ein interactive
Whiteboard oder so dann wüsste ich auch
einmal nicht wie das funktioniert. also
ich denke mir, sie glauben halt manchmal weil sie ein Handy in der Hand haben, weil sie aufgewachsen sind, dass sie das alles checken. und es
ist aber nicht so einfach und oft gibt
es ganz komische Wege auch in einer
schule, weil das bautechnisch nicht anders möglich war. Da hinten
ist das Kabel und vorne muss man es einschalten, oder? Man muss sich halt
vertraut machen mit den Gegebenheiten
\item[AS:] Verwendest du im Unterricht auch eigenes Equipment?  Eigenen Laptop, oder...?
\item[IP3:]ich hab meinen eigenen Laptop.
\item[AS:] Und warum verwendet du den?
\item[IP3:] weil, weiß ich nicht, weil das einfach mein eigener
Laptop ist und weil es praktischer ist.
\item[AS:] Würde die Schule dir einen zur Verfügung stellen für den Unterricht?
\item[IP3:] Nein, die Schule hat keinen Laptop für mich. Nein. Aber ich hab einen Stand-PC, aber es geht einfach
schneller auch, weil ich muss mich nicht einloggen, 
ich muss nicht wieder das webuntis aufmachen, oder so. Das habe ich halt da schon
offen.
\item[AS:] das heißt der private Laptops vereinfacht
die Sachen?
\item[IP3:] ja. irgendwie fahrt man sonst so zweigleisig. Es gibt aber schon Klassen, in die ich reingehe wo ich weiß ich brauche nicht
sehr viel das Notebook, wo ich meines halt dann auch nicht mitnehmen und den PC
verwendet von der schule wenn sie darum
geht was einzutragen.
aber prinzipiell habe ich ihn eigentlich
immer mit muss ich sagen auch wenn ich
ihn nicht verwenden einfach so zur
Sicherheit.
\item[AS:] abschließend vielleicht noch wenn
irgendwann mal nicht funktioniert gibt
es da entsprechende und auch
funktionierende Supportstrukturen in
der schule? ist das klar an wen du dich da wenden
musst?
\item[IP3:] Also das funktioniert bei uns nicht so gut. 
es gab früher so ein online, glaube ich,
Supportanfrageformular.
mittlerweile ist das so, so wie ich es wahrnehme, so ein ein "Zurufsystem". Und das ist, glaube ich, nicht so ideal. Ich muss aber sagen, ich bekomme das nicht so stark mit, weil solange ich mein Notebook habe und Internetverbindung. Und meine Schüler arbeiten nicht in PC-Räumen,
jetzt wo es muss auch immer wieder
Probleme halt gibt. aber das System beruht so mehr so auf zetteln hinlegen
de zuständigen Personen.
\item[AS:] Ist das vielleicht auch
ein Mitgrund warum du dein privates
Notebook verwendest?
\item[IP3:] ja ich fühle mich
halt auch sicher und ich denke zum
Beispiel da sind die Lautsprecher schon
so gut, sollten die Lautsprecher nicht
gehen habe ich zumindest die und ich stelle mich mit dem Notebook in die Mitte von der klasse
und ist es geht auch.
\item[AS:] Vielen Dank.
\end{itemize*} 
}
\section*{Interview 4} 
\texttt{
\begin{itemize*} 
\rightlinenumbers*
\modulolinenumbers[5] 
\linenumbers[0] 
\item[AS:] Einer dieser pädagogischen Grundsätze, 
wie man auch immer wieder gehört ist,
dass man die Schüler versucht dort
abholen wo sie gerade sind und in ihren
Lebenswelten zu begegnen, und das Handy
quasi ist ja aus der Lebenswelt unserer Schüler
nicht wegzudenken, das ist sozusagen fest
verankert. Welche rolle sollte deiner
Meinung nach das Handy in der schule
spielen?
\item[IP4:] Also ich bin, was die Handynutzung betrifft, auch sehr kritisch. Wenn es
darum geht YouTube Videos anzuschauen, zu
chatten, zu spielen mit dem Handy, und
damit verbringen die jugendlichen die
meiste zeit auch in den pausen zum teil
sogar im Unterricht, das heißt an und für
sich bin ich sehr dafür dass es eine
strenge Regelung gibt und dass die Handys am besten abgedreht im Rucksack sind
oder weg. Allerdings merke ich, dass die
Schüler, wenn man sie lässt, das Handy
auch sehr sinnvoll einsetzen können.
Es werden immer wieder Dinge
abfotografiert wenn es die nur einmal gibt.
das heißt sie können das abfotografieren
und das nützen und dafür lasse ich ihnen
das Handy auch. Oder ich versuche es eben
damit ich die jugendlichen dort abholen
wo sie sind ihnen zu zeigen dass man das
Handy auch sinnvoll nutzen kann.
in der ersten klasse habe ich viel
gemacht jetzt wenn es um Rechtschreibung
geht bei einer Hausübung die Dinge dann
über diese, da gibt's ja so wie ein Wörterbuch und man kann sogar durch die Wörter
sprechen und das Handy schreibt es dann
richtig das heißt die Hausübung mit
dem Handy zu kontrollieren, weil es
schneller geht natürlich als im
Wörterbuch nachzuschlagen.
also ist mit vielfältige Möglichkeiten
das sinnvoll einzusetzen, Grad in Deutsch.
Da mache ich es schon.
\item[AS:] Und was sind die gründe,
wenn ich das richtig verstanden habe,  die
gründe es nicht zu verwenden ist, dass
die Schüler abgelenkt sind?
\item[IP4:] Ja, abgelenkt. 
 das heißt
sobald das Handy natürlich wieder Medium
zur Kommunikation wird oder
zum Spielzeug wird, dann ist es noch mehr
Ablenkung.
\item[AS:] vielleicht noch eine frage:
welche fächer unterrichtet du?
\item[IP4:] Deutsch und Musik.
\item[AS:]  und wie
lange schon?
\item[IP4:] Das zehnte Jahr.
\item[AS:]  welche Einstellung hast du
zu digitalen Medien so pauschal so
grundsätzlich? verwendest du zu hause Telebanking, Online shopping, etc.?
\item[IP4:]  also ich
bin ganz altmodisch. Ich hab bis jetzt ein altes Nokia gehabt, wo ich nicht einmal ins Internet konnte. Das heißt, die die Schüler haben es  auch genossen mir
Dinge zu erklären. Das ist wieder ein Bonus
weil dann sehen sie: Ah die Frau Professor kennt sich nicht so aus und wir können ihr das erklären.
 das heißt ich nütze dann die Chance und sag: ihr
dürft das mit dem Handy machen und ihr zeigt
wie's geht oder nehmts was auf in Musik und
macht so eine  klangcollage und dann spielen
wir das über die "Bum". Das heißt ich bin da sehr stiefmütterlich.
\item[AS:] Das heiß, Telebanking oder streamen, wie Netflix....?
\item[IP4:] Nein, mach ich gar nichts. Ich kenne mich da gar nicht aus. Kann ich nicht.
\item[AS:] Und hast du den Umgang mit dem Computer in deiner
Ausbildung gelernt?
\item[IP4:] Nein.
\item[AS:] also auf der Uni war
damals gar nichts?
\item[IP4:] Nein, dazu bin ich schon zu alt.
\item[AS:] 
hast du da in dem Bereichen dann Fort- oder
Weiterbildungskurse besucht? So zu sagen an der pädagogischen Hochschule zum
Beispiel Lehrerfortbildungen zum Einsatz
von einer Lernplattform, oder so?
\item[IP4:] Nein, das mach ich auch nicht eigentlich. Das einzige was ich verwende ist
der Beamer. Eine zeit lang hab ich immer mit dem Smartboard gearbeitet. Das war super. da haben wir da an der
schule einen Workshop gehabt. Aber die
funktionieren nicht mehr das heißt wir
haben jetzt Smartboards da hängen aber
können keine spiele mehr damit machen oder Unterrichtsmaterialien verwenden. Das ist kann es nur mehr als Beamer verwenden.
\item[AS:] das heißt im Unterricht verwendest du
das Handy wie du schon erwähnt hast...
\item[IP4:] Also, ich lass die Schüler es verwenden. Und Beamer verwende ich und YouTube natürlich für
Musik ab und zu. leider das interaktiv Smartboard das funktioniert nicht mehr. unsere
Computer sind in EVA (=Eigenverantwortliches Lernen) Klassen. Da haben wir drei Laptops. Inzwischen sind es pro Klassen, glaub ich, nur mehr einer. Und der funktioniert meistens nicht. Also das würde ich gerne integrieren: Laptops in allen Klassen. die Computerarbeit, das
recherchieren um Referate zu gestalten
und so weiter. nur ex funktioniert zum
x-ten mal nicht. Wobei da auch viel an den Schülern liegt, weil die schaffen es
nicht einmal den Laptop an den
richtigen platz zurück zu stellen oder anzustecken an das richtige Kabel.
\item[AS:] Die Gefahr ist auch, dass die Schüler glauben sie kennen sich aus
und im Endeffekt kennen sie sich auch gar nicht aus.
\item[IP4:] Oft passieren dann so Dinge, wie dann
ist was nicht gespeichert worden oder es ist dann weg weil jemand anderer dort
war an dem Computer  oder sie haben es wo falsch gespeichert und dann ist es weg.
\item[AS:] Wir kommen eh nachher noch zu diesen Problematiken. Gibt 
es Phasen im Unterricht wo du  zb das
Handy oder YouTube besonders gerne
einsetzt oder gibt es Phasen wo das gar
nicht eingesetzt zum Beispiel zur Wissensüberprüfung. Setzt du da auch irgendwie
Medien ein?
\item[IP4:] da habe ich einmal, oder ein Phase hab ich gehabt, dieses Kahoot eingesetzt, weil sie das gerne gemacht haben.sonst eigentlich nicht.
\item[AS:]
aber sonst nicht in einer speziellen
Phase jetzt sondern einfach wie es halt
so gerade passt. 
\item[IP4:] Wie es gerade passt.
\item[AS:] Verwendest du auch digitale Unterrichtsmaterialien? schon
vorbereitete? also, jetzt keine Ahnung,
irgendwelche Unterrichtssequenzen wo es
zum Beispiel ein Video gibt und dann
Kontrollfragen oder Arbeitsaufträge dazu, so fertige Pakete, sag ich jetzt einmal.
\item[IP4:] Hmmm, naja zu den Deutschbüchern gibt es was dazu. Da gibt es Grammatikübungen. Das gibts Online. Die verwend ich in Deutsch. Von "Treffpunkt" gibt es etwas und von der "Deutschstunde". Das ist für die Freiarbeit, wo sie selbstständig arbeiten können.
und sonst, dass ich mir eine Powerpoint vorbereite oder
Diagramme hin projizierte, das suche ich mir
raus, oder Bildimpulse,  das schon.
Und für das smartboard hab ich früher
Sachen gemacht. 
\item[AS:] Da hat es auch fertig Unterrichtspakete gegeben?
\item[IP4:] Nein, da hat man
so leere spiele gehabt und die hat man befüllt mit Inhalten. Und das haben sie dann selber spielen können oder die
ganze Klasse. Und das war zum Beispiel zu
Wissensüberprüfung.
\item[AS:] Gibt es für dich gründe wo, digitale Unterrichtsmittel nicht einzusetzen?
\item[IP4:] In Deutsch ganz viel.  Also diese selber
schreiben  darf auf keinen Fall
verloren gehen.
natürlich wäre es ab und zu sinnvoll
im Deutschunterricht, wenn die
Kinder einen Laptop hätten und auch Dinge am Laptop schreiben könnten. Ich hab einen
Schüler der schreibt nicht gern. Oder es gibt es ab und zu Schüler die haben
dieses ADHS und überhaupt motorische
Schwierigkeiten. Da wäre es wieder wichtig, dass man sagt
 die dürfen auch am Laptop schreiben.
 \item[AS:] aber dass du sagst
es gibt in bestimmten Situationen einen Grund
digital Unterrichtsmittel bewusst nicht
einzusetzen? Außer jetzt eben
die die Motorik oder so oder in Musik zum Beispiel: Fällt dir so konkret ein, wo du sagt: Nein, da auf keinen Fall irgendwas digitale
sondern... oder gibt es das eigentlich gar
nicht in deinem Bereich?
\item[IP4:]
 wenn es
ergänzend ist stört es eigentlich nicht.
wenn es ausschließlich das wäre, also da würde ich mich dagegen wehren Ich bin auch
Montessoripädagogin und ich glaub das alles
feinmotorische, was man tut und
einordnet wichtige
Fähigkeiten sind
\item[AS:]okay danke das war auch schon der erster teil. Für den zweiten Teil, da hab ich eine Studie gelesen aus den
70er Jahren und zwar hat die
Namen "Konstanzer Wanne" die beschreibt
sozusagen den Praxisschock den Lehrer
die ihre Ausbildung gemacht haben
und voller Tatendrang in die schule
kommen und mit der wirklichen Welt konfrontiert sind und einen
richtigen Praxisschock
haben. 
jetzt ist meine frage: gibt es da auch einen Einfluss vom Lehrerkollegium in deiner
Haltung zur Verwendung von digitalen Unterrichtsmitteln? ja oder nein? Also
gibt's da grundsätzlich im Kollegen
positive Einstellung oder eine negative
Einstellung oder kannst du das deiner
Sicht nicht beurteilen?
\item[IP4:] ich glaube, dass
viele frustriert sind weil so viel nicht
funktioniert. gerade vorher hat mich eine
Kollegin angesprochen, die verzweifelt war, weil sie nirgends rein gekommen ist. sie wollen sie für ein 
Seminar melden, etwas vorbereiten.
man kommt nirgends rein, das Internet geht
nicht, WebUntis geht nicht. Also diese Frustration kriege ich mit
und da habe ich selber auch.
also das ist oft sehr ärgerlich, weil wenn da 26 Schüler sitzen
und warten und man kommt da wo nicht rein und es
funktioniert nichts. Das kenne ich.
\item[AS:]
 Aber es gibt jetzt keine
Haltung dass man sagt: also bei uns ist
das quasi so üblich dass man immer
digitaler Unterrichtsmittel einsetzt oder bei uns ist das
üblich dass man das nicht macht.
\item[IP4:] Nein, das gibt es nicht.
\item[AS;] Auch in der Fachgruppe, wenn du ein bisschen kleiner denkst. Ich habt ja doch ein großes Kollegium.
\item[IP4:] Ich glaub, dass es die meisten noch nicht verwenden.
\item[AS:]
also das ist eher eine persönliche
Haltung? Entweder man setzt es ein oder nicht und man fühlt sich keinem Druck ausgeliefert?
\item[IP4:] Ja, genau.
\item[AS:] gibt es Initiativen
von der Direktion die das beeinflussen? Zum Beispiel, dass die Direktorin sagt: wir haben ein neue
Software gekauft. Bitte nutzt das. 
\item[IP4:] 
 das
war damals bei den Smartboards. Das ist inzwischen kein Thema mehr. Die funktionieren nicht mehr. 
Ich sage oft: ich hätte gerne für
alle Klassen Laptops haben, nicht nur in den EVA-Klassen, dann heißt es: Es gibt kein Geld, oder es ist kein Thema.
\item[AS:] und auch
Initiativen vom Landesschulrat oder vom
Ministerium: Beeinflusst das irgendwie den Einsatz von digitalen
Medien oder so?
\item[IP4:]
bei einer Konferenz ist gesagt worden dass
das ja jetzt in alle
fächer einfließen soll.
ich glaube dass viele der älteren
Kollegen da ein Problem haben werden. ich bin
ja im Mittelfeld würde ich sagen aber ich kenne mich nicht sehr gut aus. und dann haben
wir ja noch viele ältere... die jungen vielleicht nicht so...
\item[AS:]  Aber als gelernte Lehrer wissen wir, wenn etwas in alle Unterrichtsfächer einfließen soll, dann heißt das vermutlich soviel wie: es passiert gar nichts.
\item[IP4:] Genau. Weil jeder glaubt, es macht eh der andere.
\item[AS:]OK das heißt wenn ich das
zusammenfassen darf: ein bisschen
herrscht schon eine negative Stimmung
 aufgrund der Infrastruktur
aber nicht grundsätzlich aufgrund des
Mediums oder des Einsatzes im Unterricht.
\item[IP4:]
Ich kann das so nicht sagen. ich glaube dass die jungen schon dem
gegenüber positiv eingestellt sind aber es gibt sicher viele sagen: das mache
ich sicher nicht.
\item[AS:] Das Problem ist aber auch, dass die
jungen in der Ausbildung sehr wenig
machen.
\item[IP4:] Aber ich weiß, dass viele der Jungen schon mit dem LMS etwas machen, wie die Beurteilungen betrifft. Das machen die Älteren nicht.
\item[AS:] Also es ist unterschiedlich.
\item[IP4:] Ja.
\item[AS:] Gut danke dann gehen wir zum
letzten teil schon. wir haben schon
darüber gesprochen und das ist auch ein
riesengroßes Thema,  dass wenn ich so digitale Unterrichtsmittel einsetze, dass sehr oft
wertvolle Unterrichtszeit verloren geht.
weil bei 20 Geräten, eines sicher nicht funktioniert, einer, kommt nicht rein, beim anderen ist der Akku
leer.... Wir kennen das.
Wie wird dein Unterricht durch die technische
Infrastruktur beeinflusst? Du hast es
schon gesagt du würdest gerne mit einem Smartboard arbeiten...
\item[IP4:] Das geht nicht.
\item[AS:] Wenn du an deinen
Unterricht denkst: findest du in der
Klasse die erforderliche technische
Infrastruktur vor und funktioniert diese
auch zuverlässig?
\item[IP4:] Nein. In manchen Klassen hätte ich ein bisserl
was, aber da funktioniert nicht alles und da bin
ich zum Beispiel wieder dankbar, wenn die Kinder ihre Handys auspacken und sich irgendwie retten können. manche
haben Internetzugang, die können dann übers Handy recherchieren, eben über
das abfotografieren funktioniert einiges, oder was aufnehmen, oder sie schicken Termine, Hausübungen.
\item[AS:] Also es fehlt ja auch etwas um den 
Unterricht nach deinen Vorstellung
umzusetzen?
\item[IP4:] Ja, genau. Wenn das alles funktionieren würde, bräuchten wir wahrscheinlich gar keine Handys.
\item[AS:] Fühlst du dich sicher im
Umgang mit dem vorhandenen technischen
Equipment, das ihr habt? Sei es jetzt Beamer, Computer, ...
\item[IP4:] Jein, mittel. Also wenn irgendwelche Probleme auftauchen stehe ich an. Wenn es so funktionieren würde wie zu Hause der Laptop, aber das geht meistens nicht so.
\item[AS:] also du würdest
dich jetzt nicht ganz sicher aber auch
nicht ganz unsicher bezeichnen?
\item[IP4:] Ja, genau.
\item[AS:] Verwendest du auch privates Equipment im Unterricht? Privates Notebook oder  Tablet?
\item[IP4:] Nein, eigentlich nicht.
\item[AS:] Falls mal etwas
nicht funktioniert: Gibt es da entsprechende
funktionierende Support Strukturen wo du das hin melden kannst?
\item[IP4:] Ich schick dann Schüler, Schülerinnen ins Konferenzzimmer und hoffe dass dort der Administrator gerade sitzt zufälligerweise.
Der eigentlich gar nicht zuständig ist, glaube ich.
Der Haupt zuständig ist, ist am Nachmittag oder zu bestimmten Zeiten nur da. Aber der hilft dann, wenn er da ist und kann und wenn er Zeit hat, aus. 
\item[AS:] Aber so richtig gut funktionierende Strukturen gibt es nicht? So eine Desktopverknüpfung, ein Meldesystem?
\item[IP4:] Nein, nein.
Und ich hab ewig gewartet, dass Drucker angesteckt werden, dass die Laptops funktionieren. Das hat dann bis Dezember, oder so, gedauert.
\item[AS:] Also da gibt es schon Luft nach oben?
\item[IP4:] Ja.
\item[AS:] Vielen Dank!
\end{itemize*} 
}

\section*{Interview 5} 
\texttt{
\begin{itemize*} 
\rightlinenumbers*
\modulolinenumbers[5] 
\linenumbers[0] 
\item[AS:] Dann würde ich gleich beginnen mit der
Einstiegsfrage es ist ja so ein Grundsatz
für uns Lehrer dass wir versuchen die
Schülerinnen und Schüler dort abzuholen
wo sie sich in ihrer Lebenswelt befinden
und das Handy ist ein
wesentlicher Bestandteil ihrer
Lebenswelt. Welche rolle sollte
deiner Meinung nach das Handy in der
schule und im Unterricht spielen?
\item[IP5:]
 Naja
vom Handy an und für sich bin ich nicht so begeistert
weil einfach die Ablenkungsgefahr zu
groß ist. es hat sicher seine Vorteile. es gibt Möglichkeiten und Methoden die man einsetzen kann auch
schon vieles ausprobiert über Quiz, wie Kahoot und solche Dinge aber
erfahrungsgemäß ist es einfach so, sobald das
Handy da ist, ist es eine Ablenkungsquelle.
\item[AS:] Welche fächer unterrichtest du? 
\item[IP5:] kaufmännische Gegenstände
hauptsächlich Betriebswirtschaft
Ausbildungsschwerpunkt Entrepreneurship, Business Behaviour.
\item[AS:] dort hast du Handys selber eingesetzt
schon in diesen Gegenständen oder eher
nicht?
\item[IP5:] bedingt. ja also es gibt gewisse
Dinge in den Handelsschulen teilweise wo man
jetzt nicht so gut ausgestattet ist im
Hinblick mit Medien an und für sich.
das heißt  in der HAK ist es prinzipiell
angenehm, dann aber der dritten klasse haben die Schüler
Notebooks und da verwende ich sie auch ab der
dritten klasse auch.
\item[AS:] Und gibt es auch keinen bedarf das Handy zu nutzen.
\item[IP5:]So ist es, also ich habe da
wenig bedarf. in der Handelsschule habe
ich es dann schon gelegentlich zum
recherchieren eingesetzt, dann in Wahrheit
ist es ja dann trotzdem eine Form von
kleinem Computer so zu sagen. Und wenn ich jetzt so COOL-Aufträge gemacht, wo ich sie in Gruppen
frei habe arbeiten lassen, 
dann habe ich es gelegentlich zugelassen
dass sie auch mit dem Handy ein bisschen
recherchieren können.
da ist jetzt das Notebook genauso eine
Ablenkungsgefahr.
allerdings in Maßen, muss ich auch dazu sagen.
\item[AS:] 
können wir vielleicht nur ein
schritt zurück, ich hab das vorher vergessen. vielleicht deine Einstellung
zu digitalen Medien: also setzt du privat Telebanking, online-Shopping, Videos
YouTube, ... ein? Was verwendest du da?
\item[IP5:]
 unglücklicherweise
sehr viel davon!
So ziemlich die gesamte Bandbreite. eBanking: ich gehe nicht so gerne auf die Bank, ich
mache das alles von zu hause, schauen oft
am Handy Kontostand an und  zu solchen Dinge. Shoppen, mit zwei kleinen Kindern sowieso am liebsten vor dem PC
\item[AS:] Nachrichten?
\item[IP5:] Zeitung lesen solche Dinge sowieso regelmäßig
\item[AS:] Fernsehfilme konsumieren
auch über Streaming?
\item[IP5:]
Nicht mit dem Handy
für die schule natürlich auch die
E-mails am Handy und natürlich auch das LMS
\item[AS:]
den Umgang
mit dem Computer hast du deine aber mit dem IPad.
Ausbildung gelernt?
\item[IP5:] Ja, auch.
\item[AS:] Auf der Uni?
\item[IP5:]ich bin
selbst in die HAK gegangen. Als ich in die HAK gegangen bin, ist das gerade aufgekommen, das erste Office-Pakte, wenn ich mich richtig erinnere, wo man dann schon mit
Excel mit Tabellenkalkulation gearbeitet hat. Das war der erste einstieg. 
ich habe aber dann zwischen natura und
und Lehramtsausbildung eine Multimedia
Ausbildung gemacht. und ich bin schon ein
bisschen medienaffin.
\item[AS:] 
 und hast doch auch Fort-
oder Weiterbildungskurse? Also hast du Seminare
besucht, die sich genau um diesen
Medieneinsatz gekümmert haben?
\item[IP5:] direkt was nicht Informatikunterricht
ist noch nicht. da hab ich hauptsächlich im
Bereich Wirtschaftsinformatik Access,
Excel und solche Dinge besucht. 
aber ich habe mich stark dafür
eingesetzt dass nächstes Jahr ein
Seminar angeboten
wo es genau um diesen sinnvolle Einsatz von Medien allgemeinen im Unterricht geht.
\item[AS:] Das heißt, du setzt digitale Unterrichtsmittel wenn
ich das jetzt ganz weit fassen kann im
Unterricht ein?
\item[IP5:] ja 
\item[AS:]und warum?
\item[IP5:] also in erster Linie weil
ich's praktisch finde. es macht viele
Dinge schon einfacher.
\item[AS:]Was macht es zum Beispiel einfacher?
\item[IP5:] also
wenn es jetzt um Unterricht geht: Wenn ich denk, allein die Visualisierung von
Dingen ist natürlich,... wir haben in der HAK in 
jedem Klassenzimmer einen PC mit
Internetverbindung und Beamer wenn ich mir das vorstelle, wie es noch in der
Lehramtsausbildung war,.. Die Ausbildung hatte
ich noch hauptsächlich mit Folien mit
Overheadfolien. es geht schon sehr viel mehr.
Natürlich auch die Gefahr, dass man zu
viel macht mit diesen Dingen früher hat
man vielleicht zwei, drei, vier Folien
gemacht. Heute hat man Foliensätze mit 20, 30 Folien
\item[AS:] Das heißt so ein repräsentatives Beispiel könnte sein?
\item[IP5:] also es sind verschiedene Dinge.
also grundsätzlich einmal die
Input Phase unterstützen durch
visualisieren. ich arbeite sehr mit Anschauungsmaterialien. Dadurch, dass wir verschiedene
Lerntypen haben. Ich versuche immer
wieder, spezielle im
betriebswirtschaftlichen Bereich, mit
Strukturen zu arbeiten, mit Bildmaterial
dass gewisse Dinge unterstützt.
\item[AS:] Videos?
\item[IP5:] Videos auch gelegentlich, wenn es passt. Es
ist dann schwierig genau zum Lehrstoff
immer passende Videos zu finden. Was ich lieber einsetze ist eher das LMS. In 
Unterrichtsgegenständen wo man dann mit freien Arbeiten im Zusammenhang mit
einer Lernplattform.
\item[AS:] also das
heißt du jetzt auch eine Lernplattform ein
\item[IP5:] ja.
\item[AS:] Und setzt du in bestimmten Unterrichtsphasen digitale Unterrichtsmittel ein? Eher zum
wiederholen oder zum einstieg ins Thema
oder ist das
über den Unterricht verteilt? in jeder
Phase sozusagen?
\item[IP5:] Also einsetzen tu ich es im Prinzip fast immer weil es in der Inputphase den Lehrvortrag
immer unterstützt durch irgendwelche
Medien, wenn es möglich ist. wo die
Schüler selber arbeiten hängt es jetzt vom
Thema ab. aber meistens halt die älteren
Schüler dann ab der dritten
Handelsakademie: die sollen dann schon selber
recherchieren, im Sinne des offenen Lernens.
\item[AS:] Du hast vorher gesagt, du setzt auch ein Lernmanagementsystem ein (LMS.AT) Verwendest du da bestehende
Unterrichtsmaterialien oder erstellst du
der Unterrichtsmaterialien auch selbst?
\item[IP5:] ich verwende sehr wenig bereits
erstellte Material.ich mache meiner
Unterrichtsvorbereitung eigentlich immer
selbst. Ich nehme maximal Anregungen her. Und ansonsten pack ich mir meine Lernpakete selbst.
\item[AS:] gibt es gründe für dich
digitale Unterrichtsmittel nicht einzusetzen? In bestimmten Situationen oder Phasen Fällt dir da etwas ein? zum Beispiel bei 
der Leistungsüberprüfung?
\item[IP5:] ich muss
jetzt ehrlich da sagen: Das hängt vom Gegenstand ab. in Betriebswirtschaft zb
setze ich eigentlich fast immer irgend
eine Variante ein, einfach um zu
untermauern. 
in Rechnungswesen verwende ich so gut
wie gar nicht. maximal wenn ich ein Hauptbuch mache. Wo ich Excel hab und das streamen kann. aber wenn ich da das nicht brauche
ist mir die altmodische Tafel das liebste.
\item[AS:] warum oder was ist der
Grund dafür? oder wo siehst
du dann den Vorteil oder den Nachteil von digitalen Unterrichtsmittel?
\item[IP5:] Ich weiß gar nicht, ob ich das als Vorteil oder Nachteil sehen kann. Gewisse
Dinge sind bei der Entwicklung an der
tafel einfach schöner darzustellen als wenn
man zu sehr bombardiert wird. wenn es
jetzt wirklich solche statische Inhalte
sind, die jetzt einfach eine Vortrag unterstützen oder Argumente unterstützen
ist das eine Sache aber erstens sind
gewisse Dinge an der Tafel viel schneller und weniger
aufwendig auch und man kann entwickeln und verändern
dann leicht ein gewisser Hinsicht
\item[AS:] Dankeschön, das war der erste block. Im zweiten Block geht es mehr um das berufliche Umfeld beziehungsweise um die
Situation an der schule. Im Jahr 1978 hat es eine Studie gegeben, die so
genannte Konstanzer Wanne". Da ist es darum
gegangen, da wurde beschrieben der
Praxisschock von Neulehrern. die sind von der Ausbildung gekommen, sind in die schule gekommen und haben dann 
sozusagen mehr oder weniger ihre
komplette Ausbildung über den Haufen
schmeißen müssen weil der Berufsalltag
ihnen ganz anders sozusagen begegnet ist.
das heißt die Idee hinter
meiner frage ist: gibt es einen
Einfluss des Lehrerkollegiums auf den
Einsatz von digitalen Unterrichtsmitteln Aus deiner Wahrnehmung?
beeinflusst zum Beispiel , wenn alle digitale
Unterrichtsmittel einsetzen im
Lehrerkollegium würdest du dich dann
sozusagen gedrängt fühlen es auch zu tun oder
auch umgekehrt
\item[IP5:] Aus meinen Beobachtungen glaube ich eher nicht. Weil es ist so, dass ein teil der
Kollegenschaft nutzt die Medien sehr
intensiv bis hin zur Beurteilung also
vom einstieg bis hin zur Beurteilung, das
ganze Klassenmanagement. Den ganzen Unterricht auf der Lernplattform LMS 
strukturiert, toll strukturiert. Es gibt welche, die sagen: Das mache ich sicher nicht. Die
verwenden einmal die Mail Funktion oder
solche Dinge.
\item[AS:]  und das hat aus deiner Wahrnehmung keinen Einfluss auf deinen Einsatz?
\item[IP5:] Ich bin einfach medienaffin. Ich bin jetzt nicht unbedingt ein digital
native, es war in meiner Jugend erst
der beginn dieser Dinge, aber es hat einfach viele Vorteile. Wenn ich allein ans
Klassenbuch denke, wo man früher Stricherl
machen musste und der Klassenvorstand
belastet war. Das geht schon einfacher jetzt.
\item[IP5:] gibt es
aus deiner Wahrnehmung im Lehrkörper
jetzt grundsätzlich eher eine positive oder eine
negative Stimmung gegenüber von 
digitalen Medien oder kann man es
vielleicht für den ganzen Lehrkörper gar nicht
so fest machen? vielleicht in der
Fachgruppe? 
\item[IP5:] Könnte ich jetzt nicht sagen. eine kollektive
Einstellung dazu würde ich nicht
erkennen. Es gibt welche die sind sehr
affin und es gibt welche die lehnen es komplett
ab.
\item[AS:] und in der Fachgruppe?
\item[IP5:] In der Fachgruppe der Wirtschaftspädagogen
kann man natürlich schon sagen das es
sicher die stärkste Verbreitung hat. Das liegt aber an unserer Schule sicher auch an der Einbettung des LMS-Teams am Standort.
\item[AS:] gibt es
von der Direktion irgendwelche
Initiativen die den Einsatz von digitalen
Medien im Unterricht fördern oder im
nicht fördern. Kannst du aus deiner
Wahrnehmung etwas erkennen?
\item[IP:] also Direktion nutzt
auch das LMS, die Lernplattform bis
zu einem gewissen grad zu informations- und 
Dokumentationszwecken. Aber es gibt
keinen zwang, keine Weisung irgendwelche
Medien einzusetzen.
\item[AS:] Also, dass man bei Konferenzen sagt: ihr
müsst alle oder ihr sollt alle, das gibt es nicht?
\item[IP5:] Nein.
\item[AS:] und vom Landesschulrat oder vom
Ministerium gibt es da aus deiner
Meinung Initiativen dass man das fördert, den Einsatz von digitalen Unterrichtsmitteln?
\item[IP5:] Vom Ministerium her hört man immer
wieder die Digitalisierung 4.0 und die Digitalisierungsinitiative. Das beobachte ich aber ich glaub, das ist eher in den Volksschulen und Mittelschulen. Das da bei uns so gedrängt wird, da habe ich nicht das Gefühl.
\item[AS:] Fällt dir in dem Bereich Schule und Umfeld noch was ein?
\item[IP5:] Ich denk, bei uns in der Handelsakademie ist es so
dass wir es von Haus aus schon aufgrund
der Ausbildung einen gewissen Medieneinsatz schon haben müssen. wir haben den eigenen Gegenstand "Wirtschaftsinformatik" und es gibt gewisse Dinge die für mich zum Beispiel schon einen Einfluss haben. Es ist einfach zum teil
auch das Budget wo jetzt sag: es fehlen
manches mal Lehrbücher und solche Dinge, dass man die nicht hat und da ist es dann schon praktisch
wenn man sie das einmal in LMS aufbaut. Man kann dann theoretisch auch Kollegen darauf zugreifen lassen, wenn man das
möchte.
\item[AS:] okay danke dann gehen wir zum letzten
teil, zur Technik und Infrastruktur. so
eine Erfahrung beim Einsatz von digitalen Unterrichtsmittel ist es ja oft dass
manche Dinge einfach nicht funktionieren
und dadurch wertvolle Unterrichtszeit
verloren geht. Also man beobachte sehr oft auch wenn jedes jedes Schulkind auch ein
eigenes Device hat und so. wenn du an
deinen Unterricht denkst findest du die
erforderliche technische Infrastruktur
vor
die du brauchst? funktioniert das
auch zuverlässig?
\item[IP5:] Absolut.  also darüber würde ich
mich nicht beschweren oder könnte ich mich
nicht beschweren. bei uns am Standort funktioniert das wirklich einwandfrei.
und ich denke, in gewisser Hinsicht
muss man flexibel sein. es kann in jedem
unternehmen auch mal eine EDV-Unterstützung ausfallen.
\item[AS:] Das heißt fehlt dir irgendetwas um deinen Unterricht nach
deinen Vorstellungen umzusetzen?
\item[IP5:] Zeit! Im Bezug auf die Technik eher nicht. Wie gesagt, ab der dritten klasse
haben wir sogar Notebookklassen.
das heißt wir haben dann Drucker in den
Klassen wo die Schüler ausdrucken können,
wir haben WLAN, 
wir haben Lehrer PCs, wir haben Beamer, wir haben Leinwände beziehungsweise
platz zum projizieren.
ich wüsste jetzt nicht, was man sonst noch
groß brauchen könnten.
\item[AS:] wenn Geld keine rolle spielen würde, würdest du dir etwas
wünschen oder eigentlich nein, es passt eh alles?
\item[IP5:] Am ehesten platz. weil wenn man solche Dinge
macht, ist es oft so, dass man in den
Klassenräumen sehr eng beinander sitzt. Es ist auch oft ein durchquetschen, wenn man helfen will. Man sieht gar nicht wirklich, was die Schüler machen. aber das ist ein
generelles Problem und  hat nicht nur was mit der
Technik an und für sich zu tun.
\item[AS:]
fühlst du dich sicher im Umgang mit dem
vorhandenen technischen Equipment? mit Computer, Beamer, Lautsprecher, Software?
\item[IP5:] ja. 
also ich interessiere mich privat auch
dafür. also mit so Späßen wie dass sie mir das LAN-Kabel raus ziehen oder dass sie mir den Monitor abstecken, das funktioniert bei mir nicht.
\item[AS:] das
heißt du hast auch keine Hemmschwelle, etwas nicht einzusetzen, weil du dich unsicher fühlst im Umgang?
\item[IP5:] Nein, überhaupt nicht. Wenn das Gerät nicht geht, versuch ich mal ein, zwei Minuten woran könnte es liegen? Alle Stecker mal testen. Und wenn nicht, dann muss es ohne gehen. Es muss bei jeder Präsentation auch ohne
gehen.
\item[AS:] verwendest du im Unterricht auch eigenes Equipment? Laptop, Tablet oder so?
\item[IP5:] eigentlich nicht, denn wir haben alles da. Ich versuche meine Daten in der
Cloud zu halten. Von daher brauche ich kein eigenes
Equipment.
\item[AS:] Und wenn mal was nicht
funktioniert: gibt es die entsprechenden
funktionierenden Supportstrukturen? Weißt du an wen du dich wenden kannst? wird es dann
auch zuverlässig abgewickelt?
das ist auch auf so ein wesentlicher
Punkt wenn Dinge nicht gemacht werden, 
wenn der Support nicht
funktioniert.
\item[IP5:] also für mein empfinden funktioniert es bei uns am Standort gut. 
ich hab das schon an verschiedenen
Standorten anders erlebt auch. natürlich
ist es so, dass in der schule die
Betreuung oft durch eine Lehrkraft
passiert. das heißt wenn der Kollege
gerade im Unterricht ist und bei einer
Schularbeit am PC geht etwas nicht,...
aber selbst da ist das so, also normal  nehme
ich mein Handy in den Unterricht nicht mit, bei einer Schularbeit oder bei der Matura hab ich es mit, um  eben den Hilfeschreie loszusetzen. 
\item[AS:] Aber es gibt
entsprechende funktionierenden
Supportstrukturen und das wird dann auch
eigentlich schnell reagiert?
\item[IP5:] Ja, es gibt zuständige und es ist bekannt wer zuständig ist. Es gibt bei uns am Standort auch eine externe Betreuung.
\item[AS:] Ich danke dir.
\end{itemize*} 
}



\section*{Interview 6} 
\texttt{
\begin{itemize*} 
\rightlinenumbers*
\modulolinenumbers[5] 
\linenumbers[0] 
\item[AS:] Gut also vielen dank, du weißt warum
es geht. Kurz bevor wir beginnen: Wie viele Jahre unterrichtet du schon?
\item[IP6:] ich bin jetzt im achten Jahr
\item[AS:] Und an der gleichen
schule?
\item[IP6:] Ich war mitverwendet an einer Neuen Mittelschule auch schon
das heißt ich unterrichtet schon das
achte Jahr in eine AHS, am Gymnasium
X war aber drei Jahren NMS Y mitverwendet und davor ein halbes Jahr an der NMS Z.
\item[AS:] Und welche fächer?
\item[IP6:] Englisch und Russisch, vorwiegend aber Englisch.
\item[AS:] es gibt ja so einen pädagogischen Grundsatz,
dass man versucht Schülerinnen und
Schüler dort abzuholen wo sie sich in
ihren Lebenswelten gerade befinden und
das Handy ist ein zentraler Bestandteil ihrer
Lebenswelt. Welche rolle sollte deiner
Meinung nach das Handy im Unterricht oder in
der schule spielen?
\item[IP6:] das ist sehr
schwierig. ich versuche den Einsatz
des Handys  zu minimieren, im Prinzip.  also für mich
sollte das Handy keine allzu große rolle
im Unterricht spielen da es zu hause schon
so eine große rolle spielt.
deswegen versuchen es eigentlich die
meisten Kollegen bei uns, habe ich jetzt
so den Eindruck, das Handy irgendwie zu vernachlässigen dann in der
schule selber. was ich mir vorstellen
kann wie man es im
Fremdsprachenunterricht einsetzen kann
wäre das die Schüler wirklich Vokabel
zum Beispiel nachschlagen über online
dictionarys. Das könnte ich mir vorstellen.
\item[AS:]  also nur gezielt
und punktuell.
\item[IP6:] Andererseits bin ich hauptsächlich in den EVA-Klassen 
eingesetzt bei uns und die haben Laptops ohnehin. das heißt das spielen die Handys ohnehin nicht so eine große rolle.
\item[AS:]  vielleicht zu deiner Einstellung: setzt du auch
Computer, Tablet oder andere Devices privat ein? und in welchem
Bereich? Telebanking, Internet oder wie?
Damit ich einen
Überblick bekomme wie du das selber
privat nutzt?
\item[IP6:] Sehr viel.
also telebanking ist auf jeden Fall mal ein Thema.  wir haben ein IPad 
zu hause das ständig benutzt wird, also ohne
das ipad geht es ohnehin nicht.
wir schauen online fern, streamen sehr viel, online shopping
sehr stark, fast nur online shopping wenn es um größere
Anschaffungen geht.
\item[AS:]  das heißt du hast doch eine Affinität zu all diesen
Devices?
\item[IP6:] Schon, ich bin zwar auch durchaus kritisch eingestellt aber wir verwenden es, da man im Prinzip gar nicht mehr ohne kann. 
\item[AS:] und hast du
den Umgang mit dem Computer in der
Ausbildung gelernt, im Studium?
\item[IP6:] nein.
\item[AS:] sondern wo?
\item[IP6:] selbst zu hause
ich bin jemand der,wenn er
etwas braucht, sich selbst damit
auseinandersetzt und sich versucht es
irgendwie selbst anzueignen so war es beim
Computer zumindest. im Studium hat das
eigentlich überhaupt keine platz gehabt.
\item[AS:] und hast du im Rahmen deiner Lehrertätigkeit Fort- und
Weiterbildungsveranstaltungen im Bereich
Computereinsatz im weitesten Sinn besucht?
\item[IP6:] Computereinsatz nicht, zu LMS hab ich etwas besucht.
\item[AS:] war das
verpflichtet oder freiwillig?
\item[IP6:] Das war verpflichtend. Das war im Rahmen des Unterrichtspraktikums.
\item[AS:] Setzt du digitale Unterrichtsmittel in deinem
Unterricht ein?
\item[IP6:]  ja, setze ich durchaus ein.
\item[AS:] Und könntest du so ein repräsentatives Beispiel geben?
\item[IP6:] Ich setze Videosequenzen zum Beispiel ein. Ich 
verwendete YouTube sehr oft. das bietet sich an. ich setze LMS
ein im Unterricht
\item[AS:] LMS hauptsächlich wofür?
\item[IP6:] Um Arbeitsblätter und 
Übungsmaterial zur Verfügung zu stellen und
Arbeitsaufträge zu geben.
ich setze auch die Cyberhomework ein, die sind in englisch sehr sehr wichtig. wir
haben ein Whiteboard an der schule, das verwende ich. Auch mit dem Beamer arbeite ich sehr viel. Um Lieder zu zeigen oder
Filmausschnitte zu zeigen oder Folien zu
zeigen.
\item[AS:] In welchen Unterrichtsphasen setzt du digitale Unterrichtsmittel ein? Geht es da eher um zum Beispiel zu einem Thema hin zu führen?
\item[IP6:] Ja, Stundeneinstieg.
\item[AS:] Oder das Wissen zu überprüften?
\item[IP6:] Wissen zu überprüfen auch. Um zu wiederholen, als Stundeneinstieg sehr oft.
eigentlich in alles Phasen. Ich kann das
jetzt nicht auf eine bestimmte Phase
beschränken.
\item[AS.] Erstellst du diese digitalen
Unterrichtsmaterialien selber oder
verwendest du bestehende Materialien
wie zum Beispiel von LMS.AT oder von Verlagen oder von sonstigen
quellen?
\item[IP6:] Ich erstelle die hauptsächlich selber.
\item[AS:] Und das sind zum Beispiel Hörübungen oder
Präsentationen oder Kontrollfragen und
das zum teil wahrscheinlich über LMS?
\item[IP6:] Ja das sind meistens solche Dinge. Ich verwende auch Videoschnitte, die ich finde und erstelle dann dazu Übungen.
\item[AS:] Gibt es Gründe für dich, digitale Unterrichtsmittel nicht einzusetzen? Situationen, wo du sagst: auf keinen Fall.
\item[IP6:] Schon, aber das hat dann meistens mit der Infrastruktur zu tun.
mittlerweile ist es recht gut bei uns an der schulen durch den Neubau und durch die Umgestaltung des Schulgebäudes
aber früher war das sehr schwierig. Also
da war nicht jede Klasse mit Beamer ausgestattet. Teilweise
funktioniert dann auch der
Internetzugang nicht.
also das sind dann die Dinge, warum ich es nicht verwende.
\item[AS:] Aber aus deiner Planung heraus würdest du jetzt nicht
sagen: Jetzt auf keinen Fall. zum
Beispiel online Test oder irgend so etwas.
\item[IP6:] Nein, das vereinfacht die Arbeit.
\item[AS:] Ok vielen dank das war
auch schon der erste Bereich zum Thema Person. Gehen wir zum zweiten teil.
da geht es um das berufliche Umfeld bzw.
die Situation in der schule. vielleicht
der Hintergrund warum ich danach nachfrage
ist, es gibt aus dem Ende der 70er Jahren eine Studie,
die den Praxisschock von Berufsanfängern
sozusagen thematisiert hat, das war die
Konstanzer Wanne".  Die
lehrenden, die mit ihrer Ausbildung
fertig geworden, sind sozusagen mit einer 
bestimmten Erwartungshaltung in die
schule gekommen und haben quasi
einen Praxisschock bekommen weil es
irgendwie ganz anders wahr als sie in
der Ausbildung gelernt haben. und das war
der Hintergrund dafür dass ich mir
gedacht vielleicht hat das
Lehrerkollegium an sich einen Einfluss
ob man digitalen Unterrichtsmittel
einsetzt oder nicht. vielleicht als erste
frage wie würdest du die Einstellung
also die grundsätzliche Einstellung des
Lehrerkollegiums zum Einsatz von digitalen Unterrichtsmittel beschreiben.
\item[IP6:] Sehr schwierig, weil es eben so unterschiedlich ist von Kollege zu
Kollege. Das hat meiner
Meinung nach schon auch mit dem
alter sehr viel zu tun.
\item[AS:] das heißt es gibt die
gesamte Bandbreite?
\item[IP6:] Genau.
\item[AS:]  und wenn du versuchst es auf die Fachgruppe
einzuschränken?
könnte man einen allgemeinen Trend erkennen oder eine
Einstellung?
\item[IP6:]  ich glaube dass die Anglisten da sehr aufgeschlossen
sind. Ich müsste jetzt überlegen ob es Fächer gibt wo die 
 weniger aufgeschlossen sind.
 \item[AS:] Und womit könnte das zusammenhängen glaubst du, dass die Anglisten mehr aufgeschlossen sind?
 \item[IP6:] vielleicht weil es sich wirklich für Fach sehr stark anbietet anbietet
und weil es speziell bei englisch sehr viele Möglichkeiten gibt. Und auch sehr viele Materialien. Ich könnte mir
vorstellen, dass das bei anderen sprachen
wie zum Beispiel russisch, bei meinem zweiten Fach, weniger verwendet wird
einfach weil, das vielleicht die Dinge die
man online finden würde
sprachlich einfach zu schwierig sind vom
Niveau her. Ich könnte mir
vorstellen dass es dadurch weniger
eingesetzt wird die Dinge die es bereits
gibt. aber bei Englisch bietet sich das
einfach sehr stark und daher denke dass
das vielleicht in der Fachgruppe häufiger zum Einsatz
kommt.
\item[AS:] Und könnte man auch sagen dass es
eine positive Stimmung gegenüber von digitalen Unterrichtsmaterialien gibt? ich habe auch schon
oft erlebt in der schule, dass die Stimmung war: Nein, das tun wir nicht.
\item[IP6:] Durchaus, doch positiv. Das englisch Kollegium ist aber auch ein
sehr junges.
\item[AS:] aus deiner Wahrnehmung nach:
gibt es auch fachspezifische
Unterschiede?
\item[IP6:] Da müsste ich überlegen. Es ist schwer zu sagen. Nein, ich glaub das ist sehr gemischt.
\item[AS:] auch über die Fachgruppen: Es gibt welche, die viel einsetzen und welche die weniger einsetzen.
\item[IP6:] Ich glaub schon.
\item[AS:] Beeinflusst die Haltung des Kollegiums
oder von deinen unmittelbaren Kollegen
mit denen du viel zu tun hast, deinen
Einsatz von digitalen Unterrichtsmitteln? So im sinne
von: nein wir setzen da nichts sein. wir
machen das alle analog?
\item[IP6:] Kaum eigentlich.
\item[AS:] oder umgekehrt zum Beispiel wenn jetzt
deine Kollegen sehr viel machen würden,
würde das deine seinem Einsatz auch
beeinflussen wenn du das überlegt?
\item[IP6:] Es kann schon sein, dass eine Kollegin oder
Kollege etwas verwendet wo ich mir dann denke,
das macht durchaus Sinn
das würde ich auch verwenden. Im negativen Sinn beeinflusst mich
das eigentlich nicht.
es könnte schon sein, dass man hin und wieder
etwas sieht was man dann auch verwendet
aber prinzipiell beeinflusst es mich eigentlich nicht.
\item[AS:] Und gibt es von der Direktion Initiativen die den Einsatz
von digital Unterrichtsmittel fördern?
ist dir da etwas bekannt?
\item[IP6:] Im Prinzip
weniger
es ist so dass wir die IKM-Testung  zum
Beispiel machen müssen. also das ist
vorgegeben. Das ist ein computerbasierter, standardisierter Test. Den die
Schüler teilweise zu hause ausfüllen
oder auch in dem EDV-Saal bei uns
gemeinsam. der muss gemacht werden.
aber ansonsten gibt es da keine Initiativen.
\item[AS:] Oder vom Landesschulrat oder vom Ministerium? Ist dir da was bekannt? Initiativen die das fördern?
\item[IP6:] eben auch nur IKM. Das ist vom
Landesschulrat aus eine Vorgabe.
\item[AS:] das heißt es gibt zum Beispiel von der Fachinspektorin für englisch oder so nicht
so eine Aufforderung oder oder
eine Bestärkung: setzt digitale
Unterrichtsmittel ein, oder.
\item[IP6:] Nein.Es gibt nur jetzt eine Initiative, die ist ganz neu und da weiß ich
selbst noch nicht genau wohin es geht
oder worauf das genau abzielt. Da mussten wir festhalten was wir schon im
Unterricht verwenden.  also da gibt es
jetzt eine Übersicht, die ist jetzt erstellt
worden von den Anglisten, wo es ziele gibt, 
also Zielvorgaben und was davon
von uns bereits umgesetzt wird. Aber da weiß ich nicht genau Bescheid leider.
\item[AS:]Danke, dann kommen wir schon zum letzten Teil, zur Technik und Infrastruktur. 
ich habe selber die Erfahrung gemacht
und ich höre dass auch immer wieder vom
Kollegen das eben beim Einsatz von
digitalen Unterrichtsmitteln oder sei
es jetzt auch mit den IPad-Klassen oder
Notebook-Klassen, dass sehr oft aufgrund von technischen
Problemen wertvolle Unterrichtszeit
verloren geht. Diese Erfahrung mache ich
auch selber. wie wird dieser Unterricht
beeinflusst durch die technische
Infrastruktur wenn du an deinem
Unterricht denkst: findest du die
Infrastruktur vor die du brauchst?
\item[IP6:] mittlerweile schon bei uns an der schule. zu
Schulbeginn ist es immer ein bisschen schwieriger. Das dauert sich immer einen Monat oder
länger bis wirklich alles funktioniert.
und man findet dann auch oft Klassen vor, das sind dann eben diese klasse wo man
selbst eigentlich nicht unterrichte, bei 
Wanderklassen jetzt vorwiegend, wo man in Klassenräumen kommt in denen man nicht so 
häufig unterrichtet,  und da ist es dann oft so, dass Dinge nicht richtig
angeschlossen sind, bis man drauf kommt
woran es liegt und so weiter, vergeht dann
wirklich sehr viel wertvolle des Unterrichts.
\item[AS:] hast du dir 
schon mal gedacht, wenn du jetzt zum
Beispiel eine bestimmte Stunde
vorbereitest, dass du sagst: ich mache das
jetzt nicht am Computer weil ich die
Befürchtung habe, dass etwas nicht
funktioniert und lasst es dann sozusagen
bleiben und machst es dann konventionell mit
Buch oder mit CD oder was auch immer?
\item[IP6:] 
mittlere nicht mehr, also nein.
mittlerweile, wenn es möglich ist mach ich 
das alles über den Computer, aber letztes
Jahr war das um einiges schwieriger da
war noch nicht jeder klasse mit Internet
ausgestattet bzw. mit Beamer ausgestattet und da musste ich dann
oft meinen Unterricht anders planen als
jetzt machen würde
also da war dann oft Einsatz nicht
möglich und daher musste ich dann Ideen
die ich hatte konnte ich einfach nicht
umsetzen, was dann schade war.
\item[AS:] Fehlt dir irgendetwas
 um deinen Unterricht nach deinen Vorstellungen
umzusetzen also jetzt von der Technik und Infrastruktur her?
\item[IP6:] Nein eigentlich momentan nicht.
\item[AS:] Wenn so ein Wunschkonzert
wäre was hättest du noch gern?
\item[IP6:]  also an
der schule selber ist es für mich
wirklich in Ordnung so wie es momentan ist. Hier an der PH für den Kurs
könnte ich mir schon noch Dinge vorstellen.
\item[AS:] Zum Beispiel? 
\item[IP6:] Hier an der PH halte ich das Aussprachepraktikum in Englisch und da hätte ich zum Beispiel gerne das
wirklich die Computerräume schon mit Kopfhörer ausgestattet sind.
ich musste da jedes mal zur Bibliothek 
gehen, die Kopfhörer ausborgen und abholen, 
sie rüberbringen. also das wäre schon eine
Sache die ich mir wünschen würde.
\item[AS:] und
in der schule: Du hast gesagt es gibt manche
Klassen oder einzelne Klassen mit
whiteboards. wäre das zum Beispiel
notwendig in jeder klasse ein Whiteboard?
\item[IP6:] Für mich nicht.
für mich reicht eigentlich ein Beamer um die Dinge zu tun, die ich vorhab. Und es ist auch oft so, dass beim Whiteboard die Software dazu nicht
gut funktioniert am Computer zu hause. Das heißt, wenn man da was zu Hause erstellt 
geht es nur sehr langsam. Deswegen hab ich die Software eigentlich
kaum eingesetzt auch in Klassen wo ich
wirklich das interaktive Whiteboard hatte.
\item[AS:] fühlst du dich sicheren Umgang mit
dem vorhandenen technischen Equipment
also mit Beamer, mit Computer, mit Lautsprecher, mit Software?
\item[IP6:] Ja ist in Ordnung, denk ich. Könnte besser sein aber ich fühl mich schon so sicher
dass ich es wirklich verwenden kann.
\item[AS:] War mal die Situation wo du 
gedacht hast: Nein ich mache das vielleicht
doch nicht weil ich mir nicht sicher
bin oder weil ich mich da nicht so gut
auskenne oder weil ich angst hab, dass das nicht
funktioniert und ich kann das Problem nicht lösen zum Beispiel?
\item[IP6:] Nein.
\item[AS:] Verwendest du auch eigenes Equipment im
Unterricht? einen eigenen Laptop oder eigenes Tablet?
\item[IP6:] mittlerweile
nicht mehr. musste ich aber teilweise in
den letzten Jahren weil eben die
Möglichkeit in der schule nicht ideal
waren. 
aber mittlerweile sind die Klassenräume so
gut ausgestattet, dass es nicht notwendig
ist.
\item[AS:] 
für die Vorbereitung zu hause vermutlich schon.
\item[IP6:] Ja.
\item[AS:] falls mal
etwas nicht funktioniert, gibt es die
entsprechenden Supportstrukturen in der
schule?
\item[IP6:] Es gibt schon Unterstützung
aber nicht jederzeit. Also hätte ich jetzt ein
Problem in einem 
Raum hätte ich niemanden der mich dabei unterstützt. 
ich könnte nach der Stunde nach Unterstützung fragen oder um Unterstützung
bitten,
aber nicht zu dem Zeitpunkt wo ich
wirklich dann dringend brauche und
danach ist es dann auch zu spät.
\item[AS:] Das heißt, da ist man auf sich selber gestellt?
\item[IP6:] Genau.
\item[AS:] da muss man auch die entsprechende
Problemlösungskompetenz selber mitbringen,  zumindest die einfachsten Dinge.
\item[IP6:] Genau, weil die Unterstützung also das
personal das wir da haben, es sind
entweder selbst Kollegen, die im Unterricht
sind, oder einfach schwer aufzufinden. Denn das Schulgebäude ist einfach so
groß, dass man dann auf die Schnelle jemand erwischen könnte.
\item[AS:] und wenn du an diese Situationen denkst,
hat das einen Einfluss darauf ob du jetzt
am Computer irgendetwas einsetzt oder nicht?
sozusagen wenn im Hinterkopf schwingt:
naja wenn etwas nicht funktioniert
habe ich eigentlich im Moment keine
Hilfestellung.
\item[IP6:] Eigentlich nicht weil
man muss dann halt flexibel sein. Das
irgendwie anders machen oder man kann es
dann irgendwie anders lösen.
oder man geht dann eigentlich bei der
Planung auch oft schon so vor, dass man
sich dann denkt: im schlimmsten Fall
mache ich das dann so oder so, dass man
eine alternative bei der Hand hat. Bei der IKM-Testung
jetzt speziell war es schon so, dass
ich das die Schüler zu hause habe
ausführen lassen weil das mit den
Kopfhörern und mit dem System auch nicht
funktioniert. also das habe ich nicht an
der schule gemacht weil ich die Stunde dafür nicht
opfern wollte. und ich bin auch
bestätigt worden, weil teilweise
zwischendurch das System abgestürzt ist zu Hause bei den Schülern oder sie die
Hörübungen nicht abspielen konnten und so weiter. das hat
mich schon beeinflusst in der 
Entscheidung, sie sollen es zu hause
machen.
\item[AS:] Eine abschließende Frage: glaubst du oder aus deiner Erfahrung oder Erzählungen von Kolleginnen und
Kollegen kann das einen Einfluss haben
digitale Unterrichtsmittel nicht einzusetzen weil wenn im Moment ein
Problem auftritt mir niemand helfen kann
und dass ich manche vielleicht denken:
dann mach ich es gar nicht.
\item[IP6:] Ich glaube schon, dass das auch vorkommt
aber wie gesagt das ist auch wieder von
Kollege zu Kollege unterschiedlich und
sosehr tauschen wir uns nicht aus im Kollegium. Und es gibt dann so kleinere Gruppen, die sich austauschen. In der kleine Gruppe glaube ich nicht dass das meine Kolleginnen beeinflusst aber wir
sind so ein großes Kollegium, da weiß man es auch schwer.
\item[AS:] Vielen Dank!
\end{itemize*} 
}

\section*{Interview 7} 
\texttt{
\begin{itemize*} 
\rightlinenumbers*
\modulolinenumbers[5] 
\linenumbers[0] 
\item[AS:] Also nochmal herzlichen dank dass du dir
zeit nimmst.bevor wir loslegen
vielleicht wie viele Jahre Unterricht
hast du schon?
\item[IP7:] ich habe früher in der
Elementarpädagogik gearbeitet. Ich unterrichte erste seit zwei Jahren im Gymnasium.
\item[AS:] ja ok und welche fächer?
\item[IP7:] französisch und Psychologie Philosophie und ich bin auch
für die Nachmittagsbetreuung zuständig.
\item[AS:] ok danke. Ein
pädagogischer Grundsatz ist ja sehr
oft, dass wir versuchen unsere
Schülerinnen und Schüler dort abzuholen
wo sie sich in ihrer Lebenswelt befinden
und das Handy ist er ein wesentlicher
Bestandteil der Lebenswelt unserer
Schülerinnen und Schüler. welche rolle
sollte deiner Meinung nach das Handy im
Unterricht spielen?
\item[IP7:] also ich verwende zum
Beispiel auch das Handy. für mich hat das Handy im Unterricht auch einen Platz. Zum Beispiel mache ich diese Kahoot. Das ist für mich eine gute Möglichkeit auch wissen auf eine sehr
sehr entspannte lustige Art und weise zu
prüfen. Da mache ich eben Quizzes, die ich selber gestalte. Es ist zum Beispiel
wenn, also früher habe ich sehr viel
Fotokopien gemacht, wenn ich in
französisch einen Text zu schreiben hatte. Das fotokopieren zum Beispiel
die Schüler dann von der tafel bzw. bei
Recherchen auch haben sie die
Möglichkeit das Handy zu verwenden.
\item[AS:] Hast du da auch negative Aspekte im
Hinterkopf?
\item[IP7:] es ist immer die Gefahr
wenn, zum Beispiel in der
Nachmittagsbetreuung, ja manche machen
auch so, es gibt so englisch Hausübungen, jetzt weiß ich nicht wie es heißt, aber sie
brauchen da ein Handy und einige, aber man
kennt diese Schüler und Schülerinnen, die nutzen das aus aus und dann sieht, dass sie wirklich spiele machen. das ist
natürlich die Gefahr, dass das
vertrauen missbraucht wird und da muss
man immer sehr sehr bewusst und sehr
aufmerksam als lehrender sein aber sonst
muss ich sagen ja habe ich gute Erfahrungen
damit gemacht.
\item[AS:] danke zu deiner Person
einmal grundsätzlich welche Einstellung
hast du zu digitalen Unterrichtsmedien
oder was verwendest du auch privat?
Telebanking, Internet, Streaming,
Online-Shopping. verwendest du das?
\item[IP7:] Nein, das verwende zum Beispiel gar nicht, da bin
ich sehr traditionell Ich bin auch nicht
mehr so ganz jung. was ich verwende ist,
ich weiß jetzt nicht, ob das auch gedacht oder bei
diesem begriff verstanden wird,
WhatsApp zum Beispiel ist für mich etwas,
das sehr hilfreich ist eben auch in der
Kommunikation wie zum Beispiel mit
meinem Bruder in Frankreich oder Skype
sozusagen verwende ich schon.
Ja Bankgeschäft auch aber Online-Shopping das machen ich nicht. Da bin ich sehr traditionell.
\item[AS:] Und den Umgang mit dem Computer hast du denn in
seiner Ausbildung gelernt damals?
\item[IP7:] Auf gar keinen Fall. Weil ich dann in 
meinem alter...  So bin ich langsam
aufgewachsen. Ich habe dann allerdings
ich habe später auch noch studiert und
da war es notwendig.
\item[AS:] Also im Studium dann schon?
\item[IP7:] Auf jeden Fall im zweiten
Berufsfeld. Im ersten noch nicht.
\item[AS:]Und hast du Fort-
oder Weiterbildungskurse hier zum
Beispiel an der PH im Bereich mit dem
computer- oder Medienumgang besucht?
\item[IP7:] also
ich bin zum Beispiel als Unterrichtspraktikantin herausgefordert worden weil
man von uns aufgaben oder von uns
aufgaben verlangt hat die natürlich damit
zu tun hatten. zum Beispiel ich denke
dieses Tevalo oder LMS. und ich bin langsam damit aufgewachsen
und ich hab auch im Kollegium zum
Beispiel
einfach so Kolleginnen und Kollegen die mich
unterstützt haben genau eingeführt haben.
\item[AS:] das waren dann im Prinzip verpflichtende Kurse oder?
\item[IP7:]Verpflichtende Kurse gab es im Rahmen
dieses Unterrichtpraktikums zum Beispiel.
\item[AS:] und du
setzt digitale Unterrichtsmittel jetzt
ganz weit gefasst im Unterricht ein?
\item[IP7:] Ja natürlich. ja
also gerade in französisch ist er sehr
wichtig um Video zu zeigen kurze
Filmabschnitte, YouTube, viele Lieder zum Beispiel.
ich weiß nicht, Powerpoint verwende ich auch
in Psychologie.
da gab es auch verschiedene didaktische
Portals die sehr unterstützend sind. Also ich verwende sehr viel aber es
ist für mich eine sehr auch didaktisch
und pädagogisch finde ich auch sehr viel
Materialien im Internet.
\item[AS:] und in welchem Unterrichtsphasen setzt du digitale
Unterrichtsmittel ein? Eher zum Hinführen zum Thema, zu Beginn oder zur
Wissensüberprüfung oder in jeder Phase?
Wenn du da schon an seinem
Unterricht denkst?
\item[IP7:] also sehr, sehr 
unterschiedlich. ja also zum Beispiel in
Psychologie wenn man bedenkt wenn es ein
gutes Video gibt, das ein Thema oder
ein gutes interview zum Beispiel mit
einem Wissenschaftler, eine
Wissenschaftlerin die für das Thema gut
passend ist. ja dann kann es zu beginn sein, je nachdem wie man das Stundenbild aufbaut. das kann aber auch als
Wiederholungen oder als Erweiterung also
es kann wirklich sehr unterschiedlich sein.
diese Quiz zum Beispiel mit kahoot das verwende ich eigentlich ausschließlich um
ein Stoff dann zu wiederholen.
spielerisch zu wiederholen.
\item[AS:] Und Leistungsüberprüfung? Zum Beispiel mit dem LMS, mit einem Onlinetest?
\item[IP7:] nein das habe ich zum Beispiel noch
nicht gemacht. ja also
Leistungsüberprüfung wenn ich die
Schüler und Schülerinnen die Möglichkeit
haben so im Laufe des Semesters
zu schauen, also in welchem Bereich
sich vielleicht noch ein bisschen mehr
leisten müssen, um zu einer bestimmten Note zu
kommen,
das schon.
Oder um  Dokumente zum Beispiel
zur Verfügung zu stellen.
\item[AS:] ja und wenn du diese
Unterrichtsmaterialien verwendest, 
erstellst du die selber oder verwendest
du bestehende zum Beispiel vom LMS oder von den Verlagen oder was auch
immer?
\item[IP7:] Also auch das ist unterschiedlich.
manche, also ich verwende selten etwas
ganz genau so wie es ist, es kann aber eine
sehr gute, sehr hilfreiche Grundlage sein. Diese OTP-Übungen, das gebe ich einfach als Hausübung, das ist dann auf freiwilliger Basis. Das  kann ich nicht überprüfen, ob die Schülerinnen und Schüler das machen. 
aber das ist gebe ich immer als Anregung um bestimmte Grammatikpunkte in französisch zu üben.
Das finde ich, ist sehr gut gemacht. sonst
wenn ich im Internet, ich kenne ein paar,
ich habe auch da zum Beispiel
Ausbildungen gemacht. es gibt ein
Franzose der in ..... eine Sprachschule
hat, und sie bieten zum Beispiel sehr viel Materialien, und das sind sehr
sehr gute Impulse aber meistens
adaptiere ich es. ich nehme nicht alles. Also ich verbringe noch Zeit aber
es gibt mir Impulse, die auch sehr unterstützend sind für meine Arbeit.
\item[AS:] Ich verstehe, also Grundlagen und Ergänzungen. Gibt es für dich auch Gründe digitale
Unterrichtsmittel nicht einzusetzen? Wenn du an eine bestimmte Situation im Unterricht
denkst, ich möchte ich eigentlich nicht weil...?
\item[IP7:] ja also zum Beispiel Powerpoint, ich hab zu Beginn sehr viel Powerpoint, weil ich immer gedacht
hab, es ist eine gute Zusammenfassung mit den Hauptbegriffen und dann hab ich bemerkt, auch aufgrund von Evaluation und
Rückmeldungen der Schülerinnen und
Schüler das wenn es ein zu viel ist, auch
nicht gut ist. ja also man muss wirklich
schauen, dass es ein gutes Gleichgewicht,
eine gute Mischung, aber ich versuche
einfach, ... oder Video ja also wenn man
nur mehr Videos zeigt dann hat es nicht mehr dieses, dieses, ... obwohl ja da merk ich 
schon, dass unsere Schüler und
Schülerinnen schon sehr viel saugen
können wenn es ein Video gibt. Aber man
muss trotzdem ein gutes Gleichgewicht
finden.
\item[AS:] Also für dich sind die gründe
jetzt nicht unbedingt spezifische
Inhalte sondern sondern eher von der
menge her.
\item[IP7:] Also von der Menge her jetzt ja also da
müsste ich nachdenken jetzt so spontan
gibt es etwas...
\item[AS:] Zum Beispiel wenn du sagst:
ich würde nie einen online Test machen, das ist mir lieber mit Papier und Pencil.
\item[IP7:] Ich würde gern einmal etwas
machen aber manchmal also wenn ich das
hier anführen kann, bräuchte ich noch ein
bisschen Unterstützung und Ausbildungen. Also
ich glaube dass ich da sicher noch gerne Einführungen brauchen würde,
damit sich das verwenden könnte. Ich glaub, das ist
für mich durchaus denkbar. Ist jetzt
natürlich, gerade im
Sprachunterricht ist auch die
Kommunikation sehr wichtig, also da sehe ich aber auch, da kann man auch Impuls geben, über das sprechen.
\item[AS:] Ok danke das war schon der erste Bereich
Person. wenn ich jetzt zum beruflichen
Umfeld und zur Schulsituation kommen
darf. Vielleicht als einstieg, also es
gibt sozusagen eine Studie die besagt dass das Lehrerkollegium einen
großen Einfluss auf das eigene verhalten
hat. Da gibt es aus dem ende der 70er
Jahren eine Studie die heißt "Konstanzer
Wanne", da wird beschrieben welchen
Praxisschock Lehrer  haben die
von der Ausbildung direkt in die schule
kommen, weil die Situation in der schule
ganz anders sich darstellen, wie es in
der Ausbildung gelehrt wurde. also daher habe ich gedacht es gibt
einen Einfluss des Lehrerkollegiums Wenn
du jetzt an dein Lehrerkollegium denkst
könntest du da eine Einstellung abschätzen
zum Einsatz von digitalen Unterrichtsmitteln. Gibt es eher eine positive
Einstellung oder eine negative? oder
kannst du das aus meiner Wahrnehmung feststellen?
\item[IP7:]
es ist bei uns ein sehr großes Kollegium und
somit ist die Antwort auch eine sehr
gemischte Antwort. Also es gibt sowohl sicher, also
wir haben auch einige lehrende die hier an der PH arbeiten und die sind meistens
sehr fortschrittlich. Und dann gibt es auch
andere die dann noch sehr vorsichtig im Umgang mit Medien sind. Das ist wahrscheinlich  die ältere Generation.
\item[AS:] Und wenn du jetzt an deine Fachgruppe denkst? Zum Beispiel an die Französischlehrer oder an die Psychologielehrer, kannst du da eine Einstellung erkennen?
\item[IP7:] Also da  hab ich  zu wenig
Erfahrung. Diese eine Kollegin in
Psychologie hat das als Beurteilungsmittel verwendet, sie hat es auch um
Dokumente in die Bibliothek den Schülern zur Verfügung zu
geben verwendet. sie war dabei schon im Vergleich zu vielen anderen in
der schule sehr fortschrittlich. Da hab ich sehr wenig Erfahrung, wenn ich jetzt mit den Lehrern meines Sohnes vergleiche,
er hat eine junge Mathematiklehrerin, die wesentlich mehr verwendet. 
auch diese Tabletklasse also ich
glaube da gibt es noch ein bisschen
noch eine gewisse Scheu. Ich hatte jetzt auch zum Beispiel, es wurde bei uns bestimmt das Handys eigentlich im Unterricht keinen platz
haben und in der pause auch nicht. also
ich gehe schon davon aus, dass es Ausnahmen gibt, aber es gibt schon eher eine Abneigung.
\item[AS:] Das heißt,
wenn ich zusammenfassen darf, sehr
stark von Einzelpersonen ab. Manche
machen es so,  manche machen so...
\item[IP7:] Manche sind strikt dagegen. Das spüre ich. Es gibt noch welche, die sind strikt dagegen.
\item[AS:] Also es gibt die gesamte Bandbreite.
\item[IP7:] Ja.
\item[AS:] Glaubst du, dass die Haltung des
Kollegiums oder die Kollegen deiner
Fachgruppe, den eigenen Einsatz
beeinflussen?
\item[IP7:] also in französisch
definitiv nicht, weil da würde ich sagen,
sie in die Lehrer eher in der älteren Generation, sind nicht viel älter als ich, aber sie sind vielleicht traditioneller. Und ich weiß, dass ich da sehr viel, ... und vielleicht habe ich dann wieder andere
so beeinflusst, weil sie gesehen haben, dass ich
da auch Materialien, zum Beispiel meine
damalig Unterrichtspraktikumsbetreuerin hat gesagt, dass
sie auch von mir sehr viel mitbekommen hat.
\item[AS:] Also du hast das Gefühl, dass es schon eine Beeinflussung gibt?
\item[IP7:] Ja, das glaube ich schon. Natürlich.
\item[AS:] Und gibt es von der Direktion irgendwelche Initiativen,
dass man jetzt digitale
Unterrichtsmittel einsetzen soll, oder
dass das gefördert wird?
\item[IP7:] Also im Moment ist es bei uns, ... wenn dann ist es nicht so eindeutig, dass
es bis zu mir gekommen ist. Es gibt im Moment
andere Schwerpunkte, wie die kollegiale
Hospitation, usw.
\item[AS:] vom Landesschulrat oder vom
Ministerium? Hast du was, oder über die Fachgruppe, 
über die Fachinspektoren,  dass es da Initiativen gibt?
\item[IP7:] Also in
meinen beiden Gegenständen also Moment
nicht so.
\item[AS:] Gut das war eigentlich schon
dieser zweite teil
und dann kommen wir zum letzten teil da
geht es um die Technik und Infrastruktur.
eine wichtige Erfahrung und die habe ich
auch selber sehr oft gemacht ist das
wenn ich digitale Unterrichtsmittel oder
auch bestimmte devices, Notebooks oder
Tablets im Unterricht einsetze, dass sehr oft mit Problemen zu kämpfen ist und
dadurch wertvolle Unterrichtszeit
verloren gehen. Das heißt die frage ist:
Wird dein Unterricht durch die technische
Infrastruktur beeinflusst? Also wenn du
jetzt an deinem Unterricht denkst:
Funktioniert alles zuverlässig und ist
alles erforderliche vorhanden in den Klassen?
\item[IP7:] Also das auf jeden Fall. Letztes Jahr
war es ganz ganz schlecht. Ich habe zum
Beispiel gewusst dass ich Psychologie in
einer klasse unterrichte in denen ich meistens eine funktionierende Infrastruktur habe.
französisch habe ich immer in diesem,
wie nennt man das, das war diese Notgebäude und da hatte ich nicht
einmal die Möglichkeit einen Film zu
zeigen. Dann habe ich eine Stunde in der
Woche, die ich in einer klasse ausgenutzt
habe.  also das war sehr sehr schwierig
bzw. eine kleinere Gruppe sind ja alle hinter
dem Computer gesessen und das
beeinflusst. ich habe zum Beispiel
gewusst ich hab ein Notprogramm. Ich versuche meistens dann, ich gestalte die Stunde
so und so im fach aus dass ich auch ohne
dieser Medien auskommen kann sollte so etwas passieren.
\item[AS:] aber wenn etwas nicht funktioniert, 
kannst du dich dann jemanden wenden?
\item[IP7:] Ja ja wir haben schon personal
aber das ist jetzt in einer
Unterrichtsstunde sehr schwierig. In dem Moment geht es nicht. Aber natürlich
gibt es bei uns Ansprechpartner.
\item[AS:] Vielen Dank!
\end{itemize*} 
}
\section*{Interview 8} 
\texttt{
	\begin{itemize*} 
		\rightlinenumbers*
		\modulolinenumbers[5] 
		\linenumbers[0] 
		\item[AS:] Herzlichen Dank, dass du
		dir Zeit für dieses interview nimmst. 
		Welchen Schultyp Unterrichtes du und welche
		fächer?
		\item[IP8:] AHS. Deutsch und Religion.
		\item[AS:] Es gibt so einen pädagogischen Grundsatz, wo man
		versucht die Schüler dort abzuholen
		in ihren Lebenswelten wo sie sich gerade
		befinden. Und aus meiner Wahrnehmung, ich
		glaube aus der allgemeinen Wahrnehmung
		heraus, ist das Handy ein totaler
		Lebensmittelpunkt für die Schüler. Jetzt zu meiner einsteigende frage: Was haltest du von Handys im
		Unterricht grundsätzlich?
		\item[IP8:] Handys im
		Unterricht setze ich eher selten ein, manchmal oder vielleicht gar
		nicht so selten. Es gibt für uns ipad-klasse. Wenn sie IPads haben ist das ipad
		natürlich praktischer. obwohl beim ipad
		oft das problem ist, dass sie nicht immer sofort ins internet
		einsteigen können,  dass das internet so
		langsam ist. während beim handy das
		internet viel schneller geht.
		wenn man jetzt etwas zu recherchieren
		haben dann ist das Handy sehr praktisch und dann setze ich es auch ein. Teilweise schon
		in Unterstufenklassen. In der
		oberstufenklasse auf jeden fall. da sind ja
		schon so weit, dass sie das auch
		einschätzen können. Aber eben
		hauptsächlich für recherche-praktiken.
		sonst wüsst ich jetzt nicht.
		\item[AS:] Gibt es situationen wo du sagst, hier hat das Handy gar keinen Platz oder
		können dass du dass du in jeder phase
		von unterricht irgendwie auch vorstellen?
		\item[IP8:] In jeder phrase vom unterricht nicht.
		also ich würde sagen wirklich nur zu
		recherchezwecken. sonst brauchen wir das
		handy eigentlich nicht.
		\item[AS:] okay. Wenn ich so
		auf die grundsätzliche persönliche
		affinität zu digitalen medien bezug
		nehmen darf: Was setzt du privat ein?
		hast du da? tablets,  handy,  online shopping?
		\item[IP8:] Natürlich, ich hab ein Handy, ein Tablet, Telebanking nutze ich.
		\item[AS:] Online-Shopping?
		\item[IP8:] Online-Shopping mache ich eher selten, mache ich nicht so gerne. Aber nicht wegen der Medien, sondern weil ich eher den Kontakt mit den leuten lieber hab, weil ich Kleidung lieber selber anprobiere als sie zu bestellen.
		Aber es ist manchmal nicht zu
		vermeiden, da es gewisse Dinge einfach
		nur im internet gibt. Die sind einfach
		leichter von da her zu besorgen.
		sonst ja,  ich recherchiere auch sehr
		viel, ivh schaue viel nach im internet, ich benutze
		meine sämtlichen WhatsApp-Gruppen. Die sind sowieso
		sehr praktisch,  auch mit Schüler.
		Also wenn ich klassenvorstand bin, sonst
		nicht. Weil sonst wird es zu privat. Aber als Klassenvorstand ist das sehr hilfreich.
		\item[AS:] Also auch soziale Medien nutzt du?
		\item[IP8:] Ja genau. Aber nur WhatsApp, Facebook nicht. Auf keinen Fall. Was noch, ich schaue mir auch etwas auf YouTube an.
		\item[AS:] Also Videostreaming?
		\item[IP8:] Ja, genau. Aber jetzt nicht so intensiv.
		\item[AS:] Also es gibt eine affinität zu digitalen
		medien?
		\item[IP8:] Das auf jeden Fall! Ohne das geht es ja gar nicht. Mail und so weiter, ohne das kann sowieso keiner leben. Mittlerweile von der Schule aus sind wir verpflichtet gewisse Produkte zu verwenden, Office 365. Man muss ja andauert was im Internet nachschauen. Sokrates,  digitales 	Klassenbuch. Es geht nicht anders.
		\item[AS:] Und den umgang mit dem
		computer, den grundsätzlichen, hast du
		den in deiner ausbildung damals gelernt?
		\item[IP8:] Eigentlich nicht. Erst später. In meiner Ausbildung habe ich meine diplomarbeit bereits auf einem
		Computer geschrieben. Das war schon ein fortschritt.
		aber dann ist das eigentlich erst im Learning-by-doing gekommen. Ich habe es gelernt
		während des arbeitens.
		\item[AS:] Und hast du da
		Fort- oder weiterbildungsveranstaltungen
		von der PH oder damals PI besucht? Im Umgang mit dem Computer?
		\item[IP8:] Ich	würde sagen einige SCHILF-Veranstaltungen an unserer schule. die eben auch teilweise verpflichtet
		waren. Ich weiß es jetzt gar nicht. zu Beginn wahrscheinlich schon. Irgendwie, wie soll ich sagen,
		irgendwie wird das automatisiert der Umgang mit dem Computer.
		\item[AS:] und	auch sehr viel autodidaktisch oder?
		\item[IP8:] Auf jeden Fall.
		\item[AS:] Und was
		sind jetzt so die gründe warum du
		digitale unterrichtsmittel im Unterricht einsetzt?
		\item[IP8:] erstens weil es
		manchmal total praktisch ist. es gibt zum
		beispiel auch auf youtube ganz tolle
		dokumentationen.
		früher habe ich VHS-Kassetten gehabt. Da musste man Aufnehmne, den Fernsehr besorgen. Das gibt es jetzt gar nicht mehr in
		der schule.
		ich setze gerne ein so lesetexte, lese
		beispiele. Also wenn jemand was gut
		vorlesen kann, damit die kinder sich was
		vorstellen können.
		in religion gibt es recht gute dokus.
		das problem ist eben immer, dass das
		internet in der schule nicht so toll funktioniert. Ich hab mir jetzt eine ganz tolle Doku in der achten
		klasse angeschaut in Religion un das war eine Katastrophe, weil dauernd das Internet blockiert war. Ich hab dann auch sogar die
		zuständige IT-Beauftragte geholt und das hat nicht funktioiniert. Das ist dann mühsam und kräftezehrend.
		aber prinzipiell ist das einfach total
		praktisch. ich möchte auch in zukunft die Schularbeiten alle in der Oberstufe mit dem 
		computer schreiben lassen, weil das
		einfach viel besser ist. Da werde ich
		nicht unbedingt unterstützt von unserem
		jetzigen herr direktor Ich hätte es schon viel länger machen
		wollen, aber sobald ich wieder eine
		fünfte habe fange ich das mit der an.
		\item[AS:] Müssen sie die Matura nicht am PC schreiben?
		\item[IP8] Leider nicht, noch nicht.
		\item[AS:] In der BMHS ist das schon so, glaube ich.
		\item[IP8:] Ja, in der AHS noch nicht. Und ich habe noch nichts gehört, dass das
		irgendwie verpflichtend wird eben weil
		die ausstattung einfach noch nicht in
		ordnung ist. Weil es so schwierig ist in einen informatiksaal rein zu kommen. Und
		dann muss man sich eben anmelden.
		schularbeiten blockieren da natürlich
		stundenlang den informatiksaal. Es gibt
		zu wenige wenn alle schularbeiten am Computer schreiben. Da wird das natürlich
		sehr kompliziert. Und das ist mühsam, wenn
		man sich das so erkämpfen muss.
		\item[AS:] In welchen
		Unterrichtsphasen setzt du digitale Unterrichtsmittel ein? Eher zum Hinführen zum Thema oder zum
		Wiederholen?
		\item[IP8:] Ja schon, zum Erarbeiten
		von gewissen dingen, ganz einfach. Also hin führen
		zum thema auch, das kann ein einstieg
		sein, 
		das kann eine vertiefung sein,
		überprüfung nicht. Schularbeiten geht jetzt noch nicht, Tests hab ich eigentlich keine.
		\item[AS:] Kontrollfragen nicht?
		\item[IP8:] Das mach ich in Deutsch und Religion eher nicht. Also ich weiß es auf jeden Fall von den anderen Fächern. Ich geb auch keine Hausübungen über LMS.at in Deutsch. Da finde ich, es wäre nicht der große Fortschritt. Es wäre vielleicht eine Spielerei für die Schüler, aber ich glaube dass für mich
		mehr Aufwand wäre, wenn ich das machen
		würde. Was ich auf jeden fall sehr schätze ist, 
		wenn die oberstufenschüler ihre Aufsätze selber dann unterschreiben am Computer.
		das ist für mich auf jeden fall
		viel leichter und einfacher zu lesen. Die
		schularbeiten dürfe eben noch nicht so geschrieben werden.
		aber sonst eben eher in der vertiefung des themas oder der erarbeitung
		eines themas, in gruppenarbeiten können sie dann auch Tablets und Handys vewenden.
		\item[AS:] Auch zu dokumentationszwecken?
		\item[IP8:] das sowieso so. Wenn präsentiert wird, wird
		heutzutage,  aber ich gebe es zwar noch
		zur wahl, plakat oder powerpointpräsentation oder andere präsentationen
		via computer, und die meisten Schüler nehmen eigentlich Powerpoint. Ein paar nehmen immer noch das plakat, weil
		man da auch wieder seine kreativität
		ausleben kann,  aber powerpoint ist
		natürlich sehr beliebt. das ist wird
		sowohl als auch, in der oberstufe fast
		nur mehr Powerpoint, in der Unterstufe auch Plakat.
		\item[AS:] Und diese digitalen
		unterrichtsmittel: erstellst du die selber
		oder verwendest du bestehende zum
		beispiel von LMS.at oder von Verlagen oder von anderen quellen?
		\item[IP8:]	Unterschiedlich. Also es gibt zum
		beispiel auch Topic und Jö haben da
		einiges,  die ganze nett sind,  also
		arbeitsblätter, die sind ganz okay. Ich
		erstelle dann auch selber etwas und geb
		den meistens einen fragenkatalog aus, den sie zu
		erarbeiten haben und anhand dessen sie dann arbeiten sollen.
		\item[AS:] Gibt es für dich gründe digitale
		unterrichtsmittel nicht einzuschätzen?
		\item[IP8:] Wenn das internet zum beispiel nicht
		funktioniert. wenn ich keine PCs zur verfügung habe.
		wenn die PC  nicht zur Verfügung stehen müssen
		wir eben am handy arbeiten.
		aber das ist halt... In der oberstufe
		geht's auf jeden fall, in der unterstufe
		ist es für mich problematisch.
		\item[AS:] Da haben nicht alle Daten...
	\item[IP8:] Genau, das ist das Problem. Der häufigste Grund ist eben, wenn das Internet nicht geht.
	\item[AS:] wir kommen dann
		nachher eh noch zur technik und Infrastrukturt aber wäre das für dich ein grund
		etwas nicht vorzubereiten, eine	digitale sequenz weil du unsicher bist, 
		dass die hardware oder die
		infrastruktur nicht funktioniert? Dass du sagst: Wer weiß ob es geht, dann mache ich es
		lieber "konventionell"?
		\item[IP8:]schon auch. 	mittlerweile, wenn ich weiß dass die
		computer auf jeden fall funktionieren, 
		powerpoint-präsentationen erstelle ich auch selber und zeige dann, dann ist es super.
		 ich habe es auch
		schon erlebt, dass das überhaupt nicht
		funktioniert, dass der beamer nicht
		gegangen ist, was auch immer, und dann war
		das immer sehr mühsam. Also irgendwie braucht man als lehrer fast immer ein
		zweites standbein. obwohl es in letzter zeit
		mit den Beamer eh funktioniert hat. Bei uns ist ja ein schulumbau, das muss man auch dazu
		sagen, und das ist natürlich auch nicht einfach. Und da kann es schon sein, gerade zum
		schulanfang weiß ich, da kann ich jetzt nicht
		damit rechnen, dass alles zu 100\% funktioniert. Da muss ich einmal warten. Dann wieder schon.
		also wenn ich weiß, also
		dieses Doku würde jetzt super in meinen
		unterricht passen und ich kann mir aber
		nicht sicher sein, dass das auch wirklich
		in dieser klasse funktioniert. manchmal
		nehme ich dann den ausweg, dass ich mir
		mit den schülern eine andere klasse suche, damit das dort funktioniert. oder
		wenn zb diese die BIFI-Testungen: Da muss man
		sich auch schon relativ vorher anmelden. Dann geht wieder irgendein computer nicht. Da brauche ich für 27 kindern fast zwei Säle, weil die sind in einem Saal gar nicht zur Verfügung. Es ist einfach manchmal mühsam, manches muss aber geschehen, wie die BIFI-Testung und anderes wo man es nicht weiß, 
		ja wahrscheinlich ist das manchmal schon ein Grund.
		\item[AS:] Gut, danke das war eigentlich schon der
		erste bereich zur person jetzt wollen wir zur
		schule gehen. 
		ich habe so eine Studie gefunden, die zwar
		nichts mit digitalen unterrichtsmittel
		zu tun hat, die heißt
		"`konstanzer Wanne"', das war Ende der 70er-Jahre. Da geht es
		darum dass sozusagen junge engagierter
		lehrer die dann in eine	schule kommen, zum Kollegium kommen sozusagen
		eine art praxisschock haben, weil
		einfach viele dinge anders laufen wie
		sie es von der ausbildung her kennen. und
		das hat mich irgendwie zu dieser annahme, gebracht dass das
		kollegium auf das eigene handeln einen
		einfluss hat. und jetzt habe ich mir gedacht,
		vielleicht hat auch der einsatz von
		digitalen unterrichtsmitteln, hat
		das kollegium auf den einsatz 
		einen einfluss. Dazu jetzt meine
		fragen. Wie würdest du die einstellung
		des Kollegiums im allgemeinen jetzt
		 gegenüber dem
		einsatz von digitalen unterrichtsmitteln sehen oder
		beschreiben?
		\item[IP8:] Sehr aufgeschlossen, würde ich sagen, auf jeden Fall. Wir haben sehr viele junge Kollegen, ich gehöre ja schon eher zu den älteren. Prinzipiell sehr aufgeschlossen. Auch bei den älteren Kollegen. Außerdem geht es gar nicht anders. Man ist ja dauernd damit konfrontiert. Wir müssen einfach aufgeschlossen gegenüber Neuen Medien sein.
		\item[AS:] Also eine positive stimmung?
		\item[IP8:] Auf jeden Fall.
		\item[AS:] 	glaubst du gibt es auch fachgruppen
		spezifische unterschiede? Kannst du das beurteilen?
		\item[IP8:] Es werden
		wahrscheinlich die mathematiker noch
		mehr tun in dieser hinsicht. Sprachenlehrer sind sowieso auch gezwungen, die
		klagen auch dementsprechend, weil eben oft
		etwas nicht geht, eine CD, die sie vorspielen wollen nicht geht. Die klagen am meisten, die Sprachlehrer. Aber die anderen Gruppen arbeiten genauso oft damit.
		\item[AS:] Also du siehst da weniger fachspezifische Unterschiede?
		\item[IP8:]Nicht so.
		\item[AS:] Eine gut Stimmung auch?
		\item[IP8:] Auf jeden Fall.
		\item[AS:] Glaubst du dass die haltung
		des kollegiums deinen Unterricht auch beeinflusst? Das heißt
		wenn in der schule eine positive
		stimmung oder haltungen gegenüber vom einsatz
		digitaler medien hat, dass das auch
		dich beeinflusst? Jetzt mehr oder oder weniger einzusetzen?
		\item[IP8:] Durchaus, ja glaube ich schon, ja. Obwohl man natürlich dann irgendwo in
		den Möglichkeiten eingeschränkt ist, und ich manchmal auch sehe, oder manchmal muss einfach etwas
		schneller gehen und dann ist vielleicht das Herkömmliche schneller als wenn ich es digital aufbereite. Aber ich denke schon, dass es so ist. Auf jeden Fall.
		Vor allem lerne ich auch
		wieder von anderen dazu, auch von jüngeren Kollegen, was die machen und das inspiriert mich dann auch dazu wieder ganz was Neues, auch digital zu probieren. Ja, das schon.
		\item[AS:] Und gibt es von der
		Direktion aus Initiativen, die den
		Einsatz von digitalen Medien im Unterricht
		fördern?
		\item[IP8:] Ja, wir haben IPad-Klassen, die sind sehr gefördert worden von der Direktion.
		\item[AS:] Und demnach ist auch die Haltung der Direktion so: IPad-Klassen, bereitet euren Unterricht
		digital inkludiert vor...
		\item[IP8:] Genau, genau.
		\item[AS:] Also da gibt
		es von der direktion schon den Wunsch und auch die Haltung.
		\item[IP8:] Das einzige wo eben ein bisschen gebremst wurde, war das schreiben der Deutschschularbeiten auf dem PC.
		\item[AS:] Und gibt es deiner Wahrnehmung nach vom Landesschulrat oder vom
		Ministerium auch initiativen die den
		einsatz fördern?
		\item[IP8:] Also vom Ministerium wüsste ich jetzt nichts, außer dass wir mit den BIFI-Testungen doch etwas genervt werden, muss man so sagen, weil das sind teilweise, ist auch vom landesschule teilweise ganz seltsame
		vorgaben.
		\item[AS:] Aber sonst Initiativen die das fördern
		oder pushen oder irgendwie versuchen auf
		den weg zu bringen ist dir nichts bekannt.
		\item[IP8:] Wüsste ich jetzt eigentlich nichts. Eigentlich würde ich mir vom Ministerium wünschen, dass da eben mehr getan wird, dass da mehr für die Infrastruktur getan wird. So wie bei uns, Schulumbau, wir haben drei Informatiksäle und jetzt heißt es in der neuen Schule sollen nur zwei Informatiksäle da sein. Also das verstehe ich irgendwie nicht. Es muss eigentlich viel mehr getan werden. Natürlich kostet das Geld, aber ich kann nicht eine digitale Schule wollen und da dann bremsen.
		\item[AS:] Da ist die
		haltung von ministerium ein bisschen so
		dass man sagt: Für die schüler "`Bring your own device"', das heißt sie erwarten
		oder hoffen dass die Schüler mit dem Laptop selber in die Schule
		kommen. Was natürlich wieder
		ganz andere probleme aufwirft.
		\item[IP8:] Einige haben so und so einen Laptop, immer mehr aber nicht alle. Das können sich auch nicht alle leisten.
		\item[AS:] Und doch dieses soziale
		ungerechtigkeit ist auch vorhanden, auch wenn man versucht das nicht wahrhaben.
		\item[IP8:] Unsere Hannah ist in eine Privatschule gegangen, da haben alle einen Laptop haben müssen in Mathematik. Bei uns wird er zur Verfügung gestellt. Wir sind eine öffentliche Schule. 
		\item[AS:] Okay, danke! Dann würde ich zum
		dritten teil schon gehen und da geht es
		um die Technik und Infrastruktur. Eine
		wesentliche Erfahrung, und die mache ich
		auch selber immer wieder, dass wenn man
		digitale unterrichtsmittel einsetzen möchte, es aufgrund von
		technischen problemen immer wieder
		wertvolle unterrichtszeit verloren geht.
		es geht einmal das nicht, das
		einloggen geht nicht, das internet geht nich. Also es ist 
		wirklich mühsam und das sage ich als Informatiker.
		Wenn du jetzt hat deinen unterricht
		denkst, du hast schon ein bisschen angesprochen:
		findest du in der klasse die
		erforderliche technische infrastruktur
		vor? Vor allem funktioniert diese auch
		zuverlässig?
		\item[IP8:] Eingeschränkt. Zu Schulbeginn ist es immer ein Problem und es ist immer wieder ein Problem. Es ist einfach schön, wenn alles funktioniert, das ist super, aber immer wieder
		passiert es, dass etwas nicht geht. Das fängt
		ja schon an, wenn ich in der Früh komme und etwas ausdrucken möchte, was ich zu Hause erarbeitet habe, un der drucker funktioniert nicht. Oder ich brauch ewig lang bis ich in diesen konferenzzimmer-computer
		einsteigen kann, weil das internet da so
		langsam ist.
		\item[AS:] Man überlegt sich das dann zweimal ob man das tut. Ich verstehe das.
		\item[IP8:] Ja, und das ist wirklich ganz, ganz mühsam. Aber es muss manchmal einfach sein und ich
		brauche die arbeitsblätter, ich brauchen die
		schularbeitsangaben und ich muss da
		schon relativ langfristig planen damit
		ich das auch wirklich, wenn ich es brauch, hab. Also schnell und spontan geht da mal gar nichts und es ist schon mühsam.
		fehlt dir etwas um den unterricht nach
		deinen vorstellungen umzusetzen?
		\item[IP8:] Das Internet ist
		zu langsam und die Geräte funktionieren nicht immer.
		\item[AS:] Und mehr Computerarbeitsplätze?
		\item[IP8:] Auch, auf jeden Fall. Das wäre auf jeden Fall notwendig, denn es wollen immer mehr in den Computersaal hinein, weil man einfach gewisse dinge am großen bildschirm besser arbeiten kann als auf dem kleinen Handybildschirm, das ist ja überhaupt kein Vergleich. IPad-Klassen sind
		auch toll, ist auch nett, wird auch von
		sehr vielen Kollegen gut angenommen. Wir haben IPad-Wägen, die man in die Klassen transportieren kann. Also auch das funktioniert.
		\item[AS:] Und in den Klassen, ist das alles vorhanden? Beamer und Computer?
		\item[IP8:] Ja, es ist in jeder Klasse ein Beamer und Computer vorhanden.
		\item[AS:] Und funktioniert das auch zuverlässig?
		\item[IP8:] Zu 80\% oder zu 70\%, schwer zu sagen.
		\item[AS:] Und würdest du noch etwas
		brauchen zum Beispiel ein interaktives
		Whiteboard oder irgendwelche speziellen
		Dinge um den Unterricht nach deinen Vorstellungen
		umzusetzen oder hast du all das was du
		brauchst?
		\item[IP8:] Wenn alles so funktionieren würde, wäre ich damit glücklich. Mit interaktivem Whiteboard habe ich noch nicht gearbeitet deswegen weiß ich nicht
		welchen Nutzen ich daraus ziehen würde, jetzt
		ob es mir gefällt oder nicht.
		\item[AS:] Und mit dem vorhandenen technischen
		Equipment, fühlst du dich das sicher im 
		Umgang, mit dem Computer, Beamer, Lautsprecher, mit
		der Software?
		\item[IP8:] Ja, schon. Ja und wenn ich als Lehrer manchmal wirkliches gerade nicht mehr weiter weiß, es gibt immer irgendeinen Schüler, der da ganz Top ist und der da weiterhilft. In jeder Klasse gibt es einen Computerfreak und der kennt sich aus, wenn etwas nicht funktioniert. Das ist super.
		\item[AS:] Also es ist für dich jetzt keine Hemmschwelle etwas nicht 
		einzusetzen, weil du dir nicht sicher bist zum
		Beispiel?
		\item[IP8:] Nein, eigentlich nicht, weil ich hab da kein Problem die Schüler zu fragen, weil die wissen das teilweise einfach besser, die wachsen damit auf und das geht dann super, wenn die mir helfen.
		\item[AS:] Und verwendest du auch eigenen Equipment? Einen eigenen Laptop, eigenes Tablet?
		\item[IP8:] Tablet hab ich eines von der Schule zur Verfügung gestellt bekommen, eigenen Laptop habe ich eher selten mit, den brauch ich eigentlich nicht. Meine eigenen Sticks, auf denen ich etwas abgespeichert habe. Aber sonst nichts.
		\item[AS:] Falls mal was nicht funktioniert, gibt es ja
		die entsprechenden funktionieren Supportstrukturen in der schule?
		\item[IP8:]  Gibt es schon, es braucht halt seine Zeit.
		\item[AS:] Also spontan, im Akutfall in der Klasse, kann einem dann keiner helfen, oder?
		\item[IP8:] Wenn nicht zufällig die zuständigen Lehrer da vor Ort sind, das ist eher selten. Nein, das braucht dann schon Zeit und wenn dann wieder mal was nicht funktioniert, dann muss man das denen melden. Es wird dann schon erledigt, manchmal haben auch die Probleme dabei, aber es dauert halt. Also wenn der Beamer wirklich kaputt ist, bis der dann repariert ist, das dauert dann schon ein paar Arbeitstage vielleicht, oder auch der Computer.
		\item[AS:] Das war es eigentlich
		schon. Vielen dank, dass du dir die zeit
		genommen hast!
\end{itemize*} 
}

\section*{Interview 9} 
\texttt{
	\begin{itemize*} 
		\rightlinenumbers*
		\modulolinenumbers[5] 
		\linenumbers[0] 
\item[AS:]gut vielen dank dass du dir die Zeit
nimmst und das interview durchführst mit mir.
Ich habe schon erklärt warum es geht.
Wir versuchen ja
als Lehrenden sehr oft unsere
Schülerinnen und Schüler dort abzuholen
wo sie sich gerade in ihren Lebenswelten
befinden und ich denke mir das Handy ist aus der
Lebenswelt unserer Schülerinnen und
Schüler nicht mehr wegzudenken. Welche
rolle sollte deiner Meinung nach das
Handy im Unterricht spielen?
\item[IP9:] Keine
\item[AS:] ja warum, bzw. warum nicht?
\item[IP9:] weil die Schüler sich so und so nicht mehr konzentrieren können seit sie das Handy haben
und das Handy ist für sie ein Objekt
wo sie einfach in ihre 
Impulse freilassen
und wenn ihnen irgendwas anstrengend ist, dann schauen sie auf das Handy.
\item[AS:]das heißt weil die Ablenkung da einfach
zu groß ist oder?
\item[IP9:]weil das Handy für sie
assoziiert ist für etwas sich abzulenken von 
etwas das anstrengend ist.
\item[AS:]das heißt du
setzt demnach auch keine Handys im
Unterricht ein?
\item[IP9:] Wenn sie keine PCs haben muss ich
sie leider einsetzen aber eigentlich
sollten sie im Unterricht nichts zu suchen haben.
\item[AS:] Du unterrichtest an einer HTL und welche fächer?
\item[IP9:] Physik, Informatik und Netzwerklabor
\item[AS:] Ok, um ein bisschen abzuschätzen wie die
persönliche Affinität zu digitalen
Medien ist würde ich gerne so ein
bisschen und das Umfeld auch die private
Nutzung fragen was setzt du privat ein
hast du da Tabletts, Handys, welche
Internetdienste, Onlineshopping, Telebanking
wie schaut das bei dir aus?
\item[IP9:] wie jeder
Handy, Computer, Notebook, Tablets aber eigentlich nur  zum
lesen aber nicht zum surfen
\item[AS:] und auch online-shopping, Telebanking, 
Videostreaming?
\item[IP9:] Ja
\item[AS:]also hast deine grundsätzlich eine
Affinität zu digitalen Medien oder
digital Devices?
setzt du digitale Unterrichtsmittel
im Unterricht ein?
\item[IP9:]was sind digitale Unterrichtsmittel?
\item[AS:] also zum Beispiel elektronische Bücher oder
eine Lernplattform oder oder
Videosequenzen oder spiele oder multiple
choice aufgaben zur Wissensüberprüfung.
\item[IP9:] In Informatik zwangsläufig: Da ist sowieso alles digital
was man da macht. Ich lasse sie selbst
Dinge erarbeiten von Webseiten. Das ist
digital. Das ganze ginge aber genauso natürlich aus einem Buch. Teilweise gibt es Dinge,
die interaktiv sind. Also zum Beispiel HTML, wo sie dann direkt daneben sehen
was der Code bewirkt. In Physik
manchmal YouTube-videos die ich ihnen zeig. Oder da gibts auch Simulationen, Animationen die man ihnen zeigen kann, wo man dann halt quasi das Experiment nicht wirklich
durchführen muss und so können sie dann auch zu
hause ausprobieren.
\item[AS:]Und warum setzt du digitale Unterrichtsmittel ein und warum machst du
das zum Beispiel nicht so wie du vorher gesagt hast 
mit einem Buch?
\item[IP9:] Das Buch kostet was. Und die, grad beim programmieren gibt es tausende Webseiten und Tutorials, die gratis sind.
\item[AS:] Also einfach
die Kostenfrage oder auch die
Verfügbarkeit wahrscheinlich?
\item[IP9:] Ja, Kostenfrage, Verfügbarkeit und manchmal ist es auch
praktischer, wenn der Code gleich mit Copy und Paste rüberkopiert werden kann. Und mit Suchfunktionen findet es man auch leichter.
\item[AS:] Hast du den Umgang mit dem PC, mit dem Computer in deiner
Ausbildung gelernt, in deinem Studium? 
\item[IP9:] Nein, schon lange vorher.
\item[AS:] In der schule noch oder autodidaktisch?
\item[IP9:] Selber.
\item[AS:] Und hast du im Bereich Computereinsatz
so allgemein auch Fort- oder
Weiterbildungskurs dann am PI oder an der PH besucht?
\item[IP9:] Ja, manchmal, meistens ein Desaster.
\item[AS:] Waren das freiwillig oder
verpflichtende Veranstaltungen?
\item[IP9:] Freiwillige.
\item[AS:] Aber grundsätzlich hast du da die Bereitschaft dich auch
weiterzubilden und entsprechende
Seminare zu besuchen?
\item[IP9:] Ich hab die Bereitschaft, mich weiterzubilden, aber ich hab nicht die Bereitschaft, Seminare zu besuchen, weil die Vortragenden meistens ziemlich schlecht sind.
\item[AS:] Wenn du dich dann weiterbildest?
\item[IP9:] ... dann selber.
\item[AS:] In
welchem Unterrichtsphasen setzt du
überwiegend digitale Unterrichtsmittel ein? So zum Einstieg oder zur Wissensüberprüfung oder in jeder Phase?
\item[IP9:] Eher zum Einstieg, zum Hinführen zum Thema, Beispiele geben.
\item[AS:] Verwendest du da auch so Kontrollfragen zur Wissensüberprüfung? Oder eher selten?
\item[IP9:]  Digital, nein.
\item[AS.]  Diese Unterrichtsmaterialien,  sind sie selber
stellt die du verwendest oder verwendest
du bestehende zum Beispiel aus Verlagen oder aus LMS.at oder aus anderen Quellen?
\item[IP9:] Digitale Unterrichtsmaterialien?
\item[AS:] Ja
\item[IP9:] Ich gebe ihnen vieles als PDF, übers LMS bekommen sie die PDFs, weil das ein Wahnsinn wäre, das auszudrucken. Sie haben es teilweise dann auch am Handy drauf, weil sie die Zettel nicht alle ausdrucken wollen. Deswegen haben sie halt das Handy im Unterricht, obwohl ich das nicht so mag. Ja das wars auch dann schon. Die Präsentationen bekommen sie auch. 
\item[AS:] Und das ist selbst erstellt?
\item[IP9:] Die Präsentationen sind selbst erstellt, die PDFs sind viel selbst erstellt aber das ist bei mir ein Sonderfall, weil wir ein Buch schreiben, jetzt können sie das auch als PDF haben. Da ist ein Drittel von mir, das heißt ein Drittel von diesen PDFs ist von mir erstellt.
\item[AS:] Und andere
quellen von Verlagen oder zu
Schulbüchern gibt sie auch zu
Zusatzmaterialien elektronisch oder so, 
verwendest du auch etwas?
\item[IP9:] Nein, das ist meistens nicht so toll.
\item[AS:] Das ist auch gleich die nächste Frage. Warum erstellst du die Materialien selber?
\item[IP9:] Naja, das ist das Buch. Weil wir es einfach mal so haben wollten, wie wir uns das vorstellen. Die anderen Materialien, also Präsentationen, es gibt ja keine fertigen Präsentationen. Dass man halt das wichtigste zusammengefasst hat.
\item[AS:] Gibt es für dich auch Gründe, 
digitalen Unterrichtsmittel nicht
einzusetzen?
\item[IP9:] Gründe, digitale Unterrichtsmittel nicht einzusetzen, wären wenn man vielleicht
ein bisschen zeit hätte, dass man die Experimente
wirklich machen kann. Wenn es die entsprechenden unterlagen gäbe, fertig ausgedruckt, dann brauchen sie das nicht digital zu haben. Da glaube ich sogar, dass es besser
funktioniert würde.
\item[AS:] Das hätte welchen Vorteil?
\item[IP9:] Ein Buch ist trotzdem noch etwas anderes als etwas, das selbst leuchtet. Da kann mir irgendwie besser arbeiten, glaube ich. Ein Buch ist auch, glaube ich, besser überdacht und kontrolliert, als irgendwelche selbst erstellten Unterlagen, von der Qualitätskontrolle her. Und ein richtiges Experiment ist trotzdem noch ganz was 
anderes als eine Simulation. Dazu hat man weder Zeit noch Geld noch Ausrüstung.
\item[AS:] Danke das war schon der erste
Bereich. jetzt würde ich gern zur schule
kommen zum schulischen Umfeld.
Ausgehend von einer Studie, die ich
gelesen habe, die heißt "`Konstanzer Wanne"', 
das ist aus dem ende der 70er Jahren, 
da geht es darum dass junge Lehrerinnen
und Lehrer von ihrer Ausbildung in die
schule kommen und quasi einen
Praxisschock bekommen, weil die 
Praxis oder die Realität sich eigentlich
ganz anders darstellt wie sie es
aus ihrer Ausbildung gewohnt waren. Und das hat mir auf die Idee gebracht, 
dass das Lehrerkollegium auf den
einzelnen Lehrer einen Einfluss haben
kann. Und ich habe mir gedacht, vielleicht hat es den auch beim Einsatz von digitalen Unterrichtsmitteln. Und deswegen einmal so die grundsätzliche frage: Wie
würdest du die Einstellung des Lehrerkollegiums bei euch an der schule
gegenüber von digitalen Medien oder digitalen Unterrichtsmitteln beurteilen? 
\item[IP9:] Eher skeptisch.
\item[AS:] Eher skeptisch?
\item[IP9:] Bei vielen eher skeptisch, weil "`Ich kenne mich eh nicht aus. Es ist ja nur zusätzlicher Aufwand und ich habe meine Sachen eh."' Teilweise aber auch vielleicht berechtigte Kritik, weil es wird dadurch ja nicht besser.
\item[AS:] Und glaubst du das es da auch
fachspezifische oder fachgruppenspezifische Unterschiede gibt? kannst du
das beurteilen.
\item[IP9:] Vielleicht ein bisschen, aber ich glaube das geht so quer durch. Überall in allen Fachgruppen gibt es Leute, die eher aufgeschlossen sind und welche, die nicht aufgeschlossen sind.
\item[AS:] Glaubst du, dass die Haltung des
Kollegiums deinen Unterricht beeinflusst
in Bezug auf den Einsatz von digitalen Unterrichtsmittel, wenn ich höre, dass da eher eine skeptische oder eine
vorsichtige Haltung ist, dass du dich dann auch eher zurück nimmst.
\item[IP9:] Nein, nein.
\item[AS:] Also glaubst du nicht, dass das einen Einfluss hat?
\item[IP9:] Nein.
\item[AS:] Glaubst du, dass es eine Einfluss auf andere haben könnte? Aus deiner Wahrnehmung heraus, wenn du an Gesprächen mit Kollegen vielleicht denkst?
\item[IP9:] Na sicher hat es einen Einfluss. Man kann das auch so sehen, das ist ja auch selbstverstärkend. Wenn einer eh nicht unbedingt was tun will und dann hört er noch fünf andere sagen "`Nein, das tun wir uns nicht an und das ist ein Blödsinn"', dann wird er sich in seiner Einstellung bestärkt fühlen.
\item[AS:] Gibt es von der
Direktion irgendwelche Initiativen die
den Einsatz von digitalen
Unterrichtsmittel  bei euch an der Schule fördern oder unterstützen oder forcieren?
\item[IP9:] Gibt es, aber extrem halbherzig. So "`Hätte ich gerne, aber wenn nichts passiert ist es auch Wurscht"'.
\item[AS:] Also da gibt es jetzt nicht wirklich einen Druck, den ein Lehrer verspürt...
\item[IP9:] Kein Druck, im Gegenteil, kommt ein Druck aus der Lehrerschaft, dann wird halt nachgegeben.
\item[AS:] Und gibt es vom Landschulrat oder vom Ministerium irgendwelche Initiativen, die bei dir angekommen sind, oder die du wahrnimmst, die den Einsatz von digitalen Unterrichtsmitteln fördern?
\item[IP9:] Ja es gibt welche, ich glaub schon, aber gerade dieses eEducation macht das ganze ein bisserl kompliziert unnötig.
\item[AS:] Und der Einfluss wird
wahrscheinlich auch nicht so groß sein, oder? Auf die anderen Kolleginnen und  Kollegen?
\item[IP9:] Nein.
\item[AS:] Gut, danke. Dann kommen wir schon zum letzten Teil, zur Technik und Infrastruktur.
Eine Erfahrung, die ich gemacht habe und
wahrscheinlich viele andere auch, dass
sehr oft beim Einsatz von digitalen
Unterrichtsmitteln und durch technische
Probleme was auch immer, sehr oft der wertvolle Unterrichtszeit
verloren geht, weil irgendetwas funktioniert immer wieder mal nicht.
Wenn du an deinen Unterricht denkst, findest
du in der klasse die erforderlichen
Infrastruktur vor, 
die du brauchst für deinen Unterricht?
\item[IP9:] Ja.
\item[AS:] Das heißt, was wäre das? Was brauchst du in deinem Unterricht?
\item[IP9:] Beamer, WLAN, Internetzugang, Computer, Lautsprecher.
\item[AS:] Und funktioniert das auch zuverlässig?
\item[IP9:] Im Großen und Ganzen schon.
\item[AS:] Fehlt dir etwas um deinen Unterricht nach deinen Vorstellungen umzusetzen? Wenn du jetzt an irgendwelche elektronischen Devices dankst? Oder, was hättest du gerne? Ein interaktives Whiteboard oder Konferenzscreen, oder?
\item[IP9:] Ich hab alles was ich brauch.
\item[AS:] Fühlst du sich sicher im
Umgang mit den vorhandenen technischen
Equipment? Alles, was du vorfindest?
\item[IP9:] Ja.
\item[AS:] Verwendest doch auch eigenes Equipment?
\item[IP9:] Ja.
\item[AS:] Warum?
\item[IP9:] Weil es manchmal praktischer ist, das eigene Notebook zu haben als die Sachen auf den anderen Computer draufzuspielen.
\item[AS:] Praktisch im Sinne von?
\item[IP9:] Im Sinne von Zeitersparnis und Bequemlichkeit. Ja, Bequemlichkeit.
\item[AS:]  Wenn mal was nicht
funktioniert,  gibt es da die
entsprechende funktionierenden Supportstrukturen in der Schule?
\item[IP9:] Ja, zwar nicht die schnellsten aber es gibt sie.
\item[AS:] Und es funktioniert auch?
\item[IP9:] Ja, im Großen und Ganzen schon.
\item[AS:] Wäre es für dich ein Kriterium, digitale Unterrichtsmaterialien nicht einzusetzen, wenn du wissen würdest, die Infrastruktur funktioniert nicht zuverlässig? Dass du sagst, ich nehme lieber das Buch mit, oder ich bereite ein Tafelbild vor, oder was auch immer. Oder ist es soweit, dass du dich da darauf verlassen kannst, dass alles funktioniert?
\item[IP9:] Ja, eigentlich kann man sich darauf verlassen.
\item[AS:] Also es gibt keinen plan "`B"', den du verbreitest für Eventualitäten.
\item[IP9:] Eigentlich nicht.
\item[AS:] Vielen Dank, dass du dir die Zeit genommen hast.
\end{itemize*} 
}

\nolinenumbers 
\end{document}
